%!TEX root = ../Thesis_example_compile_this.tex

%%%%%%%%%%%%%%%%%%%%%%%%%%%%%%%%%%%%%%%%%%%%%%%%%%%%%%%%%%%%%%%%%%%%%%%%%%%%%%%
\chapter{Latent Factor Imputation}
\label{ch:impute}
%%%%%%%%%%%%%%%%%%%%%%%%%%%%%%%%%%%%%%%%%%%%%%%%%%%%%%%%%%%%%%%%%%%%%%%%%%%%%%%

This chapter presents a novel algorithm for imputing latent factors called \gls{SGRI}. Imputation of latent factors implies a problem where all values are missing or unknown. Conditioning gridded realizations of latent factors requires assigning valid latent values at the data locations. The goal is to impute latent factors with a unique covariance structure that, when combined with inferred \gls{NMR} mapping, return the true observed values at the data locations within a tolerance. Traditionally, this problem has been approached through Gibbs sampling, though there are noted challenges in achieving stable convergence. \gls{SGRI} is a simple approach that combines \gls{SGS}, rejection sampling and exact data matching. Conditional moments are calculated with \gls{SGS}, ensuring the spatial relationships are correct and iterative rejection sampling ensures the collocated multivariate relationships are correct. Realizations of latent factors are generated and become conditioning data for gridded realizations. When mapped to observed space, the realizations reproduce non-Gaussian spatial features specified by the \gls{NMR}. A small synthetic example demonstrates imputing latent factors, conditionally simulating and mapping the latent space to observed space.

\section{Imputation Concepts}
\label{sec:impute}

Imputation is a method to ``fill in'' missing values \citep{little2019statistical}. Missing values may be uni- or multivariate and multiple mechanisms or patterns of missingness are possible. Multivariate transformations, common in modern geostatistical workflows, such as \gls{PPMT} \citep{barnett2014projection} require homotopic sampling necessitating imputation methods. Ignoring these missing values results in a loss of information or potential bias if the missingness mechanism is not random. The simplest deterministic approach to imputation is taking the global mean or median of sampled values or employing a regression model. This inference permits the use of all data in subsequent modeling however no uncertainty in the imputed values is captured.

A \textit{latent variable} is a variable that is not directly observed but is assumed related to, and can be inferred from, measured or observed variables \citep{everitt2010cambridge}. Imputation of latent factors is a unique imputation problem where all variables are missing or unsampled \citep{little2019statistical}. The latent factors are not directly observed; they are a synthetic feature of the inferred mathematical model. An example is the \gls{LMR}, which is composed of multiple latent independent random factors operating at different scales \citep{goovaerts1992factorial}. The latent factors are never measured or directly observed but characterize the observed regionalized random variable $Z(\mathbf{u})$.

Single imputation involves imputing a single value for each missing data value. \cite{little2019statistical} describe single imputation techniques as (1) mean imputation where the global mean value is substituted; (2) regression imputation where missing values are replaced by predicted value from a regression of missing variable on the observed variables; and (3) stochastic regression imputation where missing values are replaced by a predicted value plus a residual. The single imputation paradigm amounts to saying the missing values are known and constant. Though all data may be included in statistical analysis, this assumption results in incorrect uncertainty as the true values are unknown. Furthermore, data sets imputed with mean or regression imputation will not have the correct mean or variance \cite{barnett2015multivariate}. For these reasons multiple imputation techniques are preferred when characterizing imputation uncertainty is important.

Multiple imputation involves generating realizations of missing values such that imputation uncertainty can be assessed. A model of the conditional distribution of the missing values given the observed values is inferred and then stochastically sampled resulting in complete data set realizations. Multiple imputation in a geostatistical context is often a constrained problem where the imputed values must (1) reproduce underlying multivariate relationships and (2) reproduce the spatial variability of observed values\citep{barnett2015multivariate}. Collocated variables characterize the multivariate relationships and the covariance structure of the observed variables characterize the spatial variability. The conditional distribution from which the imputed values are drawn is informed by these components \citep{hadavand2023spatial}.

Simulating latent variables subject to other observations or constraints is commonly performed with a Gibbs Sampler \citep{barnett2015multivariate,emery2014simulating,silva2017multiple,arroyo2020iterative}. The Gibbs sampler is a Markov chain Monte Carlo method used to sample a multivariate distribution where direct sampling is complex but sampling marginal distributions is simple. The Gibbs sampler is practical for indirectly sampling high-dimensional distributions using univariate conditional distributions, though convergence of the algorithm is a known issue with correlated variables \citep{silva2018enhanced}.

\section{Gibbs Sampler}
\label{sec:gibbs}

\section{Sequential Gaussian Rejection Imputation}
\label{sec:sgri}