%!TEX root = ../Thesis_example_compile_this.tex

%%%%%%%%%%%%%%%%%%%%%%%%%%%%%%%%%%%%%%%%%%%%%%%%%%%%%%%%%%%%%%%%%%%%%%%%%%%%%%%
\chapter{Software}
\label{ch:software}
%%%%%%%%%%%%%%%%%%%%%%%%%%%%%%%%%%%%%%%%%%%%%%%%%%%%%%%%%%%%%%%%%%%%%%%%%%%%%%%

The following Appendix presents the software developed for the methodologies in this thesis. Programs are developed as a series of FORTRAN 90 codes with \gls{GSLIB} style parameter files \citep{deutsch1992geostatistical}. \texttt{NMROPT} is developed for \gls{NMR} parameter inference and \texttt{NMRIMP} for imputation of latent Gaussian factors. These programs together form the suite of programs for the \gls{NMR} framework. The following sections present the parameter file for each program, summarizing the key parameters. All source code is available at the authors github (\href{https://github.com/benjiharding/NMROPT}{\texttt{NMROPT}} and \href{https://github.com/benjiharding/NMRIMP}{\texttt{NMRIMP}}).


\section{NMROPT}
\label{sect:nmropt}


\begin{framed}
   \begin{lstlisting}[style=ccgParameterfile]
            PARAMETERS FOR NMROPT
            *********************

      START OF PARAMETERS:
      data.dat                          - file with data
      1 4 5 6 7 11                      - columns for dh, x, y, z, var and wt
      0                                 - normal score transform var? (0=no, 1=yes)
      -1.0e21    1.0e21                 - trimming limits
      100                               - number of unconditional realizations
      0                                 - simulation type (0=LU (<2500 data), 1=sequential)
      5841044                           - random number seed
      1 1                               - debugging level, realization to output (-1 for all)
      nmropt.dbg                        - file for debugging output
      nmropt.out                        - file for network output
      nmrwts.out                        - file for optimzed network weights
      nmrobj.out                        - file for objective function value per iteration
      nmr_                              - prefix for target/experimental output files
      3                                 - number of network layers (input to output layer)
      5 5 1                             - network layer dimensions (input + nugget to output layer)
      1 0.1                             - network wt regularization (0=none, 1=L1, 2=L2), lambda
      pool.dat                          - file with covariance structs. of Gaussian pool
      0                                 - consider factor precedence? (0=no, 1=yes)
      1  1  1  1                        - objective components: varg, ivarg, runs, npoint
      1  1  1  1                        - objective weight:     varg, ivarg, runs, npoint
      3                                 - number of indicator thresholds
      -1.28 0.0 1.28                    - Gaussian indicator thresholds
      1  1  1                           - threshold weights 
      1                                 - runs above or below threshold? (0=below, 1=above, 2=both)
      30                                - max number of runs to consider
      0                                 - runs target from file? (0=no, 1=yes)
      target_runs.out                   - runs target file
      1                                 - npoint above or below threshold? (0=below, 1=above)
      30                                - max number of connected steps to consider
      0                                 - npoint target from file? (0=no, 1=yes)
      target_npoint.out                 - npoint target file
      0.8 0.5 1.0 15 1000               - DE parameters: F, CR lo, CR hi, pop. size, its
      0.0 1.0                           - DE bounds: lower, upper
      omega.out                         - file with factor omega bounds
      1                                 - num. threads for parallel DE (1=serial, -1 for all)
      2                                 - number of experimental variogram directions
      0.0 22.5 1000 0.0 22.5 1000 0.0   - dir 01: azm,azmtol,bandhorz,dip,diptol,bandvert,tilt
      8  1000.0  500.0                  -        number of lags,lag distance,lag tolerance
      90. 22.5 1000 0.0 22.5 1000 0.0   - dir 02: azm,azmtol,bandhorz,dip,diptol,bandvert,tilt
      8  1000.0  500.0                  -        number of lags,lag distance,lag tolerance
      1                                 - number of target variogram models
      3                                 - number of target indicator variogram models
      1.0                               - IDW power for variogram optimization weighting
      999999                            - max number of exp. variogram pairs (see note)
      1                                 - standardize sill? (0=no, 1=yes)
      1    0.1                          - nst, nugget effect
      1    0.9  0.0   0.0   0.0         - it,cc,ang1,ang2,ang3
      10.0  10.0  10.0          - a_hmax, a_hmin, a_vert
      1    0.1                          - inst, nugget effect
      1    0.9  0.0   0.0   0.0         - iit,icc,iang1,iang2,iang3
      10.0  10.0  10.0         - ia_hmax, ia_hmin, ia_vert
      1    0.1                          - inst, nugget effect
      1    0.9  0.0   0.0   0.0         - iit,icc,iang1,iang2,iang3
      10.0  10.0  10.0         - ia_hmax, ia_hmin, ia_vert
      1    0.1                          - inst, nugget effect
      1    0.9  0.0   0.0   0.0         - iit,icc,iang1,iang2,iang3
      10.0  10.0  10.0         - ia_hmax, ia_hmin, ia_vert
    \end{lstlisting}
\end{framed}



\FloatBarrier
\section{NMRIMP}
\label{sect:nmrimp}


\begin{framed}
   \begin{lstlisting}[style=ccgParameterfile]
            PARAMETERS FOR NMRIMP
            *********************

      START OF PARAMETERS:
      data.dat                - file with data
      1 4 5 6 7 11            - columns for dh, x, y, z, var and wt
      0                       - normal score transform var? (0=no, 1=yes)
      -1.0e21    1.0e21       - trimming limits
      100                     - number of realizations
      5841044                 - random number seed
      1                       - debugging level
      nmrimp.dbg              - file for debugging output
      nmrimp.out              - output file with imputed realizations
      3                       - number of network layers (input to output layer)
      5 5 1                   - network layer dimensions (input + nugget to output layer)
      0                       - normalize layer inputs? (0=no, 1=yes)
      nmrwts.out              - input file with optimzed network weights
      0  1  -1.5              - consider factor precedence? (0=no, 1=yes), factor index, wt
      0  0.9  1               - seed values? (0=no, 1=yes), threshold to seed, factor index
      20                      - exclusion radius for seeded values (approx. 5-10*comp length)
      40                      - maximum previously simulated nodes
      50000 10000             - maximum iterations for step 1 and step 2
      0.1 0.01                - rejection tolerances for step 1 and step 2
      pool.dat                - file with covariance structs. of Gaussian pool
    \end{lstlisting}
\end{framed}

