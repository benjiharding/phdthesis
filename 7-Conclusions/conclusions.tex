%!TEX root = ../Thesis_example_compile_this.tex

%%%%%%%%%%%%%%%%%%%%%%%%%%%%%%%%%%%%%%%%%%%%%%%%%%%%%%%%%%%%%%%%%%%%%%%%%%%%%%%
\chapter{Conclusions and Future Work}
\label{ch:07conclusions}
%%%%%%%%%%%%%%%%%%%%%%%%%%%%%%%%%%%%%%%%%%%%%%%%%%%%%%%%%%%%%%%%%%%%%%%%%%%%%%%

The connectivity of extreme values is of great practical importance. In most real world scenarios, transfer functions are sensitive to extreme values, and correctly characterizing their two- and multi-point spatial connectivity can have significant downstream impacts. A small proportion of extreme gold grades may be the economic foundation of a mine, while an extremely low permeability facies may preclude production of a hydrocarbon reservoir. The geologic processes responsible for generating these features are not completely disorganized; strings of connected extreme values exist in nature are likely the most consequential spatial features. Traditional estimation and simulation algorithms are maximum entropy models based on multivariate Gaussian assumptions. Though practical from a dimensionality perspective, the use Gaussian \glspl{RF} smooths high-grades and tends to disconnected extreme values. These reasons motivate the development of the \gls{NMR} framework. A mixture of Gaussian \gls{RF} components may result in a highly non-Gaussian \gls{RV}. The \gls{NMR} exploits this idea while considering high-order connectivity measures from strings data and the potential spatial asymmetry between extreme highs and lows.


\FloatBarrier
\section{Summary of Contributions}
\label{sec:07contrib}

The main contributions of this thesis is the development of the \gls{NMR} framework for simulating continuous non-Gaussian spatial fields. The framework includes important contributions to quantifying non-Gaussianity from spatial data, and multiple imputation for latent Gaussian factors.

\subsection{Extreme Value Connectivity}
\label{subsec:07connect}

The connectivity of extreme values is an important metric to understand potential shortcomings of multivariate Gaussian models. Extracting multi-point connectivity measures from drill strings is not a new idea \citep{ortiz2003characterization,boisvert2007multiplepoint}, however using the $n$-point connectivity function and distribution of runs to calculate a proxy for non-Gaussianity is a contribution of this thesis. This non-Gaussianity metric can provide valuable insight into spatial regions that are not well modeled by a Gaussian \gls{RF}. These regions could be sub-domained, or simply warrant further investigation.

Extreme value connectivity is a defining feature of the \gls{NMR} framework that sets it apart from traditional multivariate Gaussian simulation algorithms. The maximum entropy characteristics of a Gaussian \gls{RF} are overcome through introduction of high-order statistics to the mixing model. It is shown that generating \gls{NMR} realizations with explicit consideration of extreme value connectivity results in spatial connectivity features that cannot be reproduced with a traditional algorithm. The $n$-point connectivity function and distribution of runs provide straightforward access to high-order statistics without the use of a training image.

\subsection{Network Model of Regionalization}
\label{subsec:07nmr}

The \gls{NMR} is a material contribution of this thesis. It is an expansion of the \gls{LMR} concept for univariate spatial modeling. Both models construct a random function as a combination of independent, standard normal factors, each with its own basic covariance model, each operating at different spatial scales. The key difference with the \gls{NMR} is the combination of factors is not limited to be linear. Activation of the latent factors with a novel \gls{MPL} function prior to the linear combination permits the spatial structure of certain features to be imparted on the tails of the distribution. Traditionally, the \gls{LMR} is limited to four or five factors while there is no practical limit with the \gls{NMR}. This flexibility permits mixing of latent covariance structures in creative ways, achieving complex modeling goals. Structures with various anisotropies and orientations can be combined reproducing non-stationary features. A key concept of the \gls{NMR} is that a mixture of Gaussian factors can be highly non-Gaussian.

The \gls{NMR} is parameterized by $2 \cdot (M+1)$ parameters: $M+1$ weights and $M+1$ power law exponents. Inference of these parameters is an inverse problem. The spatial features of the observed data are known, however the parameters that map the latent space to the observed space are unknown. Framing this as an optimization problem permits use of multi-component objective function. It is through this objective function that the \gls{NMR} gains the ability to reproduce statistics beyond the second order. Two- and multi-point statistics are introduced directly through the optimization process. The parameters of the network are optimized such that the \gls{NMR} output has the correct continuous and indicator variogram models, $n$-point connectivity function, and distributions of runs. The \gls{NMR} can be thought of as an approximation of the non-linear mapping function between the latent factor space and the observed data space. The approach is highly flexible with no limitations on the number of latent factors or objective function components. The \gls{NMR} is an important contribution to geostatistics, particularly for continuous simulation of high-order connectivity features.

\subsection{Sequential Gaussian Rejection Imputation}
\label{subsec:07sgri}

\Gls{SGRI} is a novel algorithm for multiple imputation of latent factors within the \gls{NMR} framework, and is a practical alternative for the Gibbs sampler. The algorithm is designed to overcome Gibbs sampler convergence issues with spatially correlated data \citep{silva2018enhanced}. \Gls{SGRI} combines elements of \gls{SGS} and rejection sampling to impute spatially correlated latent variables that reproduce data observations when mixed through the \gls{NMR}. The algorithm directly samples all univariate conditional distributions of the $M$-dimensional latent distribution until the mapped value is within a specified tolerance. Sampling the conditional distributions ensures variogram reproduction. Next, each sampled value is iteratively perturbed until the mapped value matches the observed value. The algorithm rejects samples anytime they do not improve the solution. Iterative refinement of the solution ensures the collocated multivariate relationships are correct while still honouring the spatial covariance structure of each factor.

The algorithm shows stable convergence when both constrained and unconstrained, correctly reproducing factor variogram models and collocated multivariate relationships. \Gls{SGRI} is a notable contribution concerning latent imputation and could be adapted (with an appropriate mapping function) to truncated Gaussian categorical modeling techniques such as hierarchical truncated pluri-Gaussian simulation.

\subsection{Simulation of Continuous High-Order Features}
\label{subsec:07hosim}

The complete \gls{NMR} framework for the continuous simulation of high-order spatial features consists of the \gls{NMR} forward model and the \gls{SGRI} algorithm. This framework is the primary contribution of this thesis. The \gls{NMR} methodology is developed to overcome the maximum entropy characteristic of multivariate Gaussian \glspl{RF}. The \gls{NMR} permits the simulation of continuous variables with connectivity features beyond the second order. These multi-point spatial connectivity features are critical when characterizing the continuity of extreme values. A key differentiator from existing high-order simulation methodologies \citep{mustapha2011hosim}, is the \gls{NMR} framework does not require a training image. All high-order statistics are extracted from strings of drillhole data and embedded in the parameterization of the latent-observed mapping function. The order of these statistics is practically limited by drillhole length, however conceptually the $n$-point connectivity function can consider any $n$. Another key component is that the \gls{NMR} framework leverages the simplicity of the Gaussian \gls{RF} model for generating non-Gaussian spatial structures. This spatial Gaussian mixture model concept simplifies implementation of the framework as both unconditional and conditional Gaussian realization can be generated with any algorithm.

Imputation of latent factors proceeds after defining all parameters of the \gls{NMR}. Considering the mapping function during imputation ensures the latent factors used for conditioning have the correct spatial features. Finally, mapping of the gridded latent realizations to observed space with the \gls{NMR} ensures the models reproduce all target two- and multi-point statistics. Practical considerations for inference of \gls{NMR} parameters are discussed in detail, including (1) latent factor design, (2) \gls{MPL} activation function parameterization, (3) objective function design, (4) potential non-uniqueness of the solution, and (5) checking of forward model outputs. Detailed discussion is also provided concerning the practical aspects of latent imputation, including the minimum acceptance criteria for imputed realizations. The \gls{NMR} framework is shown to effectively characterize multi-point high-grade connectivity features in a range of two- and three-dimensional examples, with the ability to adapt to highly non-stationary domains. The framework is demonstrated on a dataset from producing underground mine, where the non-Gaussian \gls{NMR} realizations show up to a 7\% improvement in contained metal over a conventional \gls{SGS} workflow in high-grade stopes. Though a modest improvement, improving forecasts of local high-grade resources is of great practical importance.

\subsection{Spatial Outlier Detection}
\label{subsec:07outlier}

An algorithm is presented to address the spatial component of outlier detection. Traditionally, the mining industry has employed graphical outlier detection techniques that neglect the spatial neighbourhood of potential outliers. The spatial correlation of geoscience data is pertinent information for describing a samples degree of outlierness. The algorithm considers the relationship of each sample within a local neighbourhood plus its probability density from a fitted \gls{GMM}. The idea is that true outliers should be either sufficiently different from their spatial neighbours, be from a low density region of the \gls{GMM}, or a combination of both. The algorithm also considers the data configuration through an area of influence. Extremely high- or low-grade samples in sparse data regions pose risk of overestimation and the outlier score is related to area of influence. A drawback of the approach, and others from Section \ref{subsec:02tools}, is that a subjective threshold must be selected to delineate inliers and outliers. Selecting this threshold may be challenging in some instances though practice shows clustering of outliers are generally apparent. The algorithm is shown to effectively identify both outliers and extreme values in one- and two-dimensional examples. The algorithm is not intended to be a replacement for existing methodologies, but an additional tool in the practitioners' toolbox. Best practice dictates identification of outliers with an ensemble of methodologies.

\FloatBarrier
\section{Limitations}
\label{sec:07limit}

Despite the developments made in this these, limitations exist with the proposed methodologies. The following sections discuss the limitations of the \gls{NMR} framework and latent variable imputation.

\subsection{Network Model of Regionalization}
\label{subsec:07nmr2}

A limitation of the \gls{NMR} approach is that the final models reproduce linear features. The model has no capacity to adapt to non-linear features as the objective function is characterized by two-point statistics and linear sequences of \gls{MPS}. Limiting \gls{MPS} to \gls{1D} sequences is a practical trade off for the use of a \gls{TI}, however this precludes generation of curvilinear features. One could conceivably generate curvilinear features with a sufficient number of latent factors at appropriate orientations, however the objective function components remain linear. This is a limitation compared to other high-order simulation algorithms. Another challenge is though a mixture of Gaussian \glspl{RF} can be non-Gaussian, the mixture may experience destructured extreme values depending on the conditioning data. Structuring the extreme values can be controlled through the objective function components to a degree, however this may not be possible with sparse data.

Another limitation of the \gls{NMR} methodology is the necessity of numerical inversion. The outputs of the forward model are known, however the mapping parameters that yield the correct forward outputs are unknown. Inverse problems are generally ill-posed, or lack a unique solution. This non-uniqueness means the solution is sensitive to initial parameter choices like the latent Gaussian pool, $\pmb{\omega}$, and $\pmb{a}$ parameters. Potential exists for the algorithm to converge on a solution that is numerically reasonable but geologically unreasonable. The practitioner must make initial parameter choices to ensure the optimization process explores an appropriate solution space. Potential exists for conflicting objective function components, or components with confounding relationships. There may be uncertainty in the objective components if they are derived from experimental statistics. It falls upon the practitioner to ensure a carefully crafted objective function, suitable conversion criteria, and appropriate regularization through prior geologic knowledge and parameter bounds.

Numerical inversion is an iterative process and typically computationally expensive. The speed of \gls{NMR} convergence is negatively correlated with the number of available data. As each iteration of the inversion generates a newly parameterized network, experimental variograms must be recalculated each time. Though the paring of data is only performed one, updating each lag of each variogram is computationally expensive. This is somewhat counteracted by parallelization of \gls{DE} however algorithm run times may be significant with greater than 10000 data.

% \begin{enumerate}
%     \item Still linear, no curvilinear features
%     \item uncertainty in objective components, relationship to number of data
%     \item conflicting objectives
%     \item lots of hyper parameters
%     \item choice of Gaussian pool
%     \item might be hard to get away from Gaussianity with strong conditioning
%     \item MPS are 1D
%     \item slow, variogram is expensive calculation repeatedly
%     \item solution non-unique
%     \item inverse considerations in general
% \end{enumerate}

\subsection{Latent Imputation}
\label{subsec:07impute}

Though the \gls{SGRI} algorithm shows stable convergence in many scenarios, a limitation of the methodology is the somewhat ``brute-force'' approach to imputation. Sequentially sampling the conditional distributions can lead to scenarios where imputed values at one data location yield local conditional \glspl{CDF} at another location that do not permit convergence. This is practically overcome by resetting all imputed values in a neighbourhood about the non-converged location and resimulating, though the algorithm simply relies on a sufficient number of iterations to overcome this issue. For these reasons, the rejection sampling approach may necessitate a significant number iterations for convergence at each data location. Similar to the \gls{NMR} workflow, conversion may be slow with many data.

\Gls{SGRI} is a practical alternative to the Gibbs sampler paradigm for imputation. However, the current implementation is not general, and requires a fully parameterized \gls{NMR} to impute latent variables. Overall, this is not a limitation of the methodology, but a consideration for adaptation to other workflows featuring latent imputation.

% \begin{enumerate}
%     \item SGRI not general, requires NMR for mapping, lots of iterations
%     \item imputation requires restarts to prevent non-convergence
%     \item iteration heavy - not really a problem
% \end{enumerate}

\FloatBarrier
\section{Future Work}
\label{sec:07fwork}

Future work should address the known limitations the proposed methodology. The \gls{NMR} framework poses challenges as the solution space is non-unique and necessitates a numerical approach. It is possible that another algorithm performs better than \gls{DE} concerning parameter inference. \Gls{DE} is chosen due to its ease of implementation and widespread use in engineering problems, though another heuristic algorithm may perform well. Regardless of the chosen optimization algorithm, future work could improve the computational efficiency of both the \gls{NMR} and \gls{SGRI} algorithms. An efficient approximation of the experimental variogram could drastically speed up \gls{NMR} parameter inference. As the \gls{NMR} objective function may contain any number of components, one could explore additional components such as directional asymmetry, or third order spatial moments from $[\mathbf{h}_{1}, \mathbf{h}_{2}]$ triplets \citep{lauzon2020sequential}.

Another area of future research is exploration of more complex network structures. The current network architecture is simple and there is no interaction of latent factors before the output layer. It is possible a more complex, or classical multi-layer perceptron architecture could capture even more complex, non-Gaussian, or non-linear spatial features.

% \begin{enumerate}
%     \item somethign other than DE
%     \item overall efficiency of both algorithms
%     \item deeper dive into more complex network structures
%     \item additional objective components? \cite{lauzon2020sequential}
% \end{enumerate}

