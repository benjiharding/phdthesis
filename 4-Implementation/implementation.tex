%!TEX root = ../Thesis_example_compile_this.tex

%%%%%%%%%%%%%%%%%%%%%%%%%%%%%%%%%%%%%%%%%%%%%%%%%%%%%%%%%%%%%%%%%%%%%%%%%%%%%%%
\chapter{Network Implementation}
\label{ch:04implement}
%%%%%%%%%%%%%%%%%%%%%%%%%%%%%%%%%%%%%%%%%%%%%%%%%%%%%%%%%%%%%%%%%%%%%%%%%%%%%%%

This chapter presents the implementation details of the \gls{NMR} introduced in Chapter \ref{ch:03framework}. The network is the first component of \gls{NMR} simulation workflow and operates in conjunction with \gls{SGRI}, introduced in Chapter \ref{ch:05impute}. Inference of network parameters, $\theta$, is an inverse problem as only the true data values are known. A mapping function, $\mathcal{F}_{\theta}$, is therefore required to map the unknown latent space to the known observed space. The function is parameterized such that a pool of unconditional, latent Gaussian factors reproduce the known values and have the desired spatial features when mapped with $\mathcal{F}_{\theta}$. As the latent space is unknown, and ultimately a synthetic feature of the \gls{NMR} model, it is free to contain any number of components with any covariance structure. This flexibility permits creativity with latent factor design; combining latent covariance structures in unique ways allows a mixture of univariate Gaussian distributions to possess non-multivariate Gaussian spatial features. Practical implementation details include considerations for design of the network and the latent Gaussian pool as well as sensitivities associated with these model parameters. A small synthetic example demonstrates latent factor design, network parameter inference and non-uniqueness properties of the \gls{NMR}. This example highlights the network component of the complete \gls{NMR} workflow and is carried forward into Chapter \ref{ch:05impute} to illustrate the second imputation component.

\FloatBarrier
\section{Network Design}
\label{sec:networkd}

The \gls{NMR} is not a true neural-network, rather a model of regionalization inspired by neural-network structure. The network consists of an input layer where the number of latent factors determines the number of input nodes, a single ``hidden layer'' of the same dimension, and an output layer with a single node.  The single output node makes the \gls{NMR} univariate (analogous to the \gls{LMR}) however there is no reason it could not be extended to the multivariate case (analogous to the \gls{LMC}). However, with geospatial data, one commonly considers extreme values in the univariate context; it is not immediately clear what constitutes a multivariate extreme.

\subsection{Architecture}
\label{subsec:arch}

Network architecture is controlled by the number of latent factors in the Gaussian pool. The current \gls{NMR} implementation restricts the architecture of the network to an input layer, a single hidden layer and a univariate output layer. The input and hidden layer both contain $M$ nodes. Figure \ref{fig:example_nmr} shows an example network configuration with $M=4$ input latent factors, $\{ Y_{1}, \dots, Y_{4}\}$. The hidden layer of the network if effectively a transformation layer, transforming the latent Gaussian values to unitless activation values. The network activation function $\phi(\dots)$ is a modified power-law function; the following subsection presents the details. The network output $Z_{1}$ is then a weighted linear combination of the activated latent factors.

\begin{figure}[htb!]
    \centering
    \includegraphics[scale=0.7, trim={0 1.5cm 0 1.5cm}, clip=true]{./0-Figures/04-Ch4/example_nmr.png}
    \caption{Schematic representation of the \gls{NMR} with $M=4$ latent factors. $\phi(\dots)$ is a modified power-law activation function. }
    \label{fig:example_nmr}
\end{figure}

The generalized forward pass through the network is defined as:
\begin{align}
    \label{eq:fpass1}
    \mathbf{x} & = \sum_{m=1}^{M}a_{m} \cdot \phi_{m}(\mathbf{y}_{m}, \omega_{m}) \\
    \label{eq:fpass2}
    \mathbf{z} & = G^{-1}\left( F_{X}\left(\mathbf{x} \right)\right)
\end{align}

Where $\mathbf{x}$ is an intermediary activation vector, $a_{m}$ is a weight applied to factor $m$, $\phi_{m}(\mathbf{y}_{m}, \omega_{m})$ is the modified power-law activation function with exponent $\omega$ applied to factor $m$, $F_{X}$ is the \gls{CDF} of $\mathbf{x}$ and $G^{-1}$ is the inverse of the Gaussian \gls{CDF}. Alternatively, Equation \ref{eq:fpass1} can be rewritten in matrix notation as:
\begin{equation}
    \mathbf{x} = \phi\left( \mathbf{Y} \right) \mathbf{a}^{T}
    \label{eq:fpass3}
\end{equation}

Where $\mathbf{a}$ is a row vector of factor weights and $\phi\left(\mathbf{Y} \right)$ is an $ndata$ x $M$ matrix of activated latent factors. The raw output of the network is units of activation that must be transformed to Gaussian space, necessitating the normal score transform. With the given architecture, the \gls{NMR} requires inference of $2 \cdot M$ parameters: a $1$ x $M$ dimensional vector of factor weights and a $1$ x $M$ dimensional vector of power-law exponents $\omega$. These $2 \cdot M$ network parameters are inferred through stochastic optimization.

\subsection{Activation Function}
\label{subsec:activation}




\begin{enumerate}[noitemsep]
    \item Architecture
    \item Gaussian transform
    \item Activation function

\end{enumerate}


\FloatBarrier
\section{Latent Factor Design}
\label{sec:factord}

\begin{enumerate}[noitemsep]
    \item Conceptual geologic model
    \item Indicator asymmetry
    \item Connectivity of extreme values
    \item Precedence and weighting
\end{enumerate}


\FloatBarrier
\section{Parameter Inference}
\label{sec:paraminfer}

\begin{enumerate}[noitemsep]
    \item Differential evolution
    \item Objective function
    \item Constraints and precedence

\end{enumerate}


\FloatBarrier
\section{Implementation Details}
\label{sec:implementd}

\begin{enumerate}[noitemsep]
    \item Number of data
    \item Number of uncond. reals
    \item Variogram approximation
\end{enumerate}


\FloatBarrier
\section{Sensitivity and Validation}
\label{sec:valid}

\begin{enumerate}[noitemsep]
    \item Non-uniqueness
\end{enumerate}


\FloatBarrier
\section{Discussion}
\label{sec:discuss04}