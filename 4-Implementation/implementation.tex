%!TEX root = ../Thesis_example_compile_this.tex

%%%%%%%%%%%%%%%%%%%%%%%%%%%%%%%%%%%%%%%%%%%%%%%%%%%%%%%%%%%%%%%%%%%%%%%%%%%%%%%
\chapter{Network Implementation}
\label{ch:04implement}
%%%%%%%%%%%%%%%%%%%%%%%%%%%%%%%%%%%%%%%%%%%%%%%%%%%%%%%%%%%%%%%%%%%%%%%%%%%%%%%

This chapter presents the implementation details of the \gls{NMR} introduced in Chapter \ref{ch:03framework}. The network is the first component of \gls{NMR} simulation workflow and operates in conjunction with \gls{SGRI}, introduced in Chapter \ref{ch:05impute}. Inference of network parameters, $\theta$, is an inverse problem as only the true data values are known. A mapping function, $\mathcal{F}_{\theta}$, is therefore required to map the unknown latent space to the known observed space. The function is parameterized such that a pool of unconditional, latent Gaussian factors reproduce the known values and have the desired spatial features when mapped with $\mathcal{F}_{\theta}$. As the latent space is unknown, and ultimately a synthetic feature of the \gls{NMR} model, it is free to contain any number of components with any covariance structure. This flexibility permits creativity with latent factor design; combining latent covariance structures in unique ways allows a mixture of univariate Gaussian distributions to possess non-multivariate Gaussian spatial features. Practical implementation details include considerations for design of the network and the latent Gaussian pool as well as sensitivities associated with these model parameters. A small synthetic example demonstrates latent factor design, network parameter inference and non-uniqueness properties of the \gls{NMR}. This example highlights the network component of the complete \gls{NMR} workflow and is carried forward into Chapter \ref{ch:05impute} to illustrate the second imputation component.

\FloatBarrier
\section{Network Design}
\label{sec:04networkd}

The \gls{NMR} is not a true neural-network, rather a model of regionalization inspired by neural-network structure. The network consists of an input layer where the number of latent factors determines the number of input nodes, a single ``hidden layer'' of the same dimension, and an output layer with a single node.  The single output node makes the \gls{NMR} univariate (analogous to the \gls{LMR}) however there is no reason it could not be extended to the multivariate case (analogous to the \gls{LMC}). However, with geospatial data, one commonly considers extreme values in the univariate context; it is not immediately clear what constitutes a multivariate extreme.

\subsection{Architecture}
\label{subsec:04arch}

Network architecture is controlled by the number of latent factors in the Gaussian pool. The current \gls{NMR} implementation restricts the architecture of the network to an input layer, a single hidden layer and a univariate output layer. The input and hidden layer both contain $M$ nodes. Figure \ref{fig:example_nmr} shows an example network configuration with $M=3$ input latent factors, $\{ Y_{0}, \dots, Y_{3}\}$. By convention, the $0^{th}$ factor is the nugget effect. The hidden layer of the network if effectively a transformation layer, transforming the latent Gaussian values to unitless activation values. The network activation function $\phi(\dots)$ is a \gls{MPL} function; the following subsection presents the details. The network output $Z_{1}$ is then a weighted linear combination of the activated latent factors.

\begin{figure}[htb!]
    \centering
    \includegraphics[scale=0.7, trim={0 1.5cm 0 1.5cm}, clip=true]{./0-Figures/04-Ch4/example_nmr.png}
    \caption{Schematic representation of the \gls{NMR} with $M=3$ latent factors. $\phi(\dots)$ is the \gls{MPL} activation function. }
    \label{fig:example_nmr}
\end{figure}

The generalized forward pass through the network is defined as:
\begin{align}
    \label{eq:fpass1}
    \mathbf{x} & = \sum_{m=0}^{M}a_{m} \cdot \phi(\mathbf{y}_{m}, \omega_{m}) \\
    \label{eq:fpass2}
    \mathbf{z} & = G^{-1}\left( F_{X}\left(\mathbf{x} \right)\right)
\end{align}

Where $\mathbf{x}$ is an intermediary activation vector, $a_{m}$ is a weight applied to factor $m$, $\phi(\mathbf{y}_{m}, \omega_{m})$ is the \gls{MPL} activation function with exponent $\omega$ applied to factor $m$, $F_{X}$ is the \gls{CDF} of $\mathbf{x}$ and $G^{-1}$ is the inverse of the Gaussian \gls{CDF}. The weights $\{ a_{0}, \dots, a_{M}\}$ are constrained to be greater than or equal to zero and reflect the relative importance of each factor in the mapping function. Alternatively, Equation \ref{eq:fpass1} can be rewritten in matrix notation as:
\begin{equation}
    \mathbf{x} = \phi\left( \mathbf{Y} \right) \mathbf{a}^{T}
    \label{eq:fpass3}
\end{equation}

Where $\mathbf{a}$ is a row vector of factor weights and $\phi\left(\mathbf{Y} \right)$ is an $ndata$ x $M+1$ matrix of activated latent factors. The raw output of the network is units of activation that must be transformed to Gaussian space, necessitating the normal score transform. With the given architecture, the \gls{NMR} requires inference of $2 \cdot (M+1)$ parameters: a $1$ x $M+1$ dimensional vector of factor weights and a $1$ x $M+1$ dimensional vector of power-law exponents $\omega$. These $2 \cdot (M+1)$ network parameters are inferred through stochastic optimization and discussed in Section \ref{sec:04paraminfer}.

\subsection{Activation Function}
\label{subsec:04activation}

The goal of the \gls{NMR} is to parameterize the arbitrary mapping function, $\mathcal{F}_{\theta}$, between the latent and observed spaces. The form of this function is not obvious, hence the use of a network based approach as a function approximation. Given the complex spatial features we wish to capture in the final models, a polynomial of degree greater than one is useful. Suppose a linear activation function is used. As the forward pass through the network is a weighted, linear combination of the inputs, a linear activation function (or simply $c \cdot \mathbf{y}$ where $c$ is a constant) results in a linear output or single order polynomial \citep{sharma2020activation}. To achieve a non-linear network output, one must introduce a non-linearity in the form of an activation function. Real data commonly contain non-linearly separable features and a non-linear activation function permits projection of these features onto a non-linear feature space \citep{dubey2022activation}.

In the traditional \gls{ML} context, the constraint of differentiability is placed on activation functions due to use of the back-propagation algorithm \citep{rojas1996backpropagation}. The \gls{NMR} is only inspired by the structure of a neural-network and its parameters are ``learned'' through gradient-free stochastic optimization. This gradient free approach, and a strength of the \gls{NMR}, permits the use of virtually any activation function, differentiable or not. The \gls{NMR} activation function is a \gls{MPL} function with the form:
\begin{equation}
    \phi \left( \mathbf{y} \right) =
    \begin{cases}
        \mathbf{y}^{\omega},           & \text{if $\mathbf{y} > 0$} \\
        \mathbf{y}^{\frac{1}{\omega}}, & \text{if $\mathbf{y} < 0$} \\
    \end{cases}
    \label{eq:power}
\end{equation}

Where $\omega$ is a trainable parameter. The magnitude of $\omega$ allows the activation function to emphasize certain regions of the latent distribution. If $\omega = 1$, the activation is linear and $\phi \left( \mathbf{y} \right)=\mathbf{y}$. If $\omega$ is less than one the function takes on a concave shape that emphasizes low values and mutes the influence of high values. When $\omega$ is greater than one, the opposite is true. The function takes on a convex shape that emphasizes high values and mutes the influence of low values. Figure \ref{fig:power_activation} (left) shows the relationship between $\phi \left( \mathbf{y} \right)$ and $\mathbf{y}$ for various values of $\omega$. As the magnitude of $\omega$ increases, the activation function becomes steeper above zero, and flatter below zero. The magnitude of the high values are increased exponentially, and low values are muted significantly. This non-linear amplification allows the network to embed high-grade features of latent factors in the mapping function $\mathcal{F}_{\theta}$. Low grade features are embedded in opposite fashion.

\begin{figure}[htb!]
    \centering
    \includegraphics[scale=0.7, max size={\textwidth}{\textheight}]{./0-Figures/04-Ch4/power_activation.png}
    \caption{The \gls{MPL} (left) and scaled \gls{MPL} (right) activation function for various values of $\omega$ and input $\mathbf{y} \in [-5,5]$. The scaled activation uses $\xi = G^{-1}(0.999) \approx 3.09$. }
    \label{fig:power_activation}
\end{figure}

It is notable that the \gls{MPL} activation has three inflection points: at -1, 0, and 1. The inflections exist as $\pm 1^{\omega} = \pm 1$ and $0^{\omega} = 0$, regardless of $\omega$. No negative effects have been observed related to these inflections, however the points at -1 and 1 may be adjusted by introducing a scaling factor, $\xi$, to the \gls{MPL} activation function:
\begin{equation}
    \phi \left( \mathbf{y}, \xi \right) =
    \begin{cases}
        \xi \cdot \left(\frac{\mathbf{y}}{\xi}\right)^{\omega},           & \text{if $\mathbf{y} > 0$} \\
        \xi \cdot \left(\frac{\mathbf{y}}{\xi}\right)^{\frac{1}{\omega}}, & \text{if $\mathbf{y} < 0$} \\
    \end{cases}
    \label{eq:power_scale}
\end{equation}

The scaling parameter $\xi$ allows a predetermined inflection point to be set while retaining the linear nature of the function when $\omega = 1$. Figure \ref{fig:power_activation} (right) shows the scaled \gls{MPL} activation function where $\xi = G^{-1}(0.999) \approx 3.09$ and $G^{-1}$ is the inverse of the Gaussian \gls{CDF}. $\xi$ effectively controls the threshold at which dampening or emphasizing a factor occurs at. When $\omega > 1$ and $0 < y < \xi$, the \gls{MPL} reduces the value of $y$ slightly, similar to an opportunity-seeking risk perspective \citep{eidsvik2015value}, saying ``we are only interested in values of $y > \xi$.'' When $\omega < 1$ and $0 < y < \xi$, the \gls{MPL} increases the value of $y$ slightly, similar to a risk-averse perspective, saying ``we are only interested in values of $y < \xi$.'' The opposite relationships hold true when $y < 0$. It is not immediately clear when the \gls{MPL} should be scaled, however if the practitioner notices artifacts related to the activation function, $\xi$ can be tuned. The range of $\omega$ for each latent factor is an important component of latent factor design, discussed in Section \ref{sec:04factord}. Constraining $\omega_{m} > 1$ enforces higher-grade features of the $m^{th}$ factor in the mapping. Conversely, constraining $\omega_{m} < 1$ enforces lower-grade features of the $m^{th}$ factor in the mapping. Values of $\omega$ are practical within $[0.25, 4.0]$ and reflect the relative influence of on the high and low values of each latent factor.

% \begin{enumerate}[noitemsep]
%     \item Architecture
%     \item Gaussian transform
%     \item Activation function
%     \item Recall kriging is linear, but results can be highly non-linear
% \end{enumerate}

\FloatBarrier
\section{Latent Factor Design}
\label{sec:04factord}

Latent factor design is a critical component of the \gls{NMR} workflow. The latent factors form the pool from which the \gls{NMR} draws structure from. The univariate Gaussian factors are mixed with the goal of generating a spatial distribution that is univariate Gaussian, but not multivariate Gaussian. The mixture distribution can only contain features, or combination of features, present in the Gaussian pool. Therefore, it is critical that any feature required in the final model be present in the pool. The Gaussian pool is similar to the concept of factorial kriging \citep{goovaerts1997geostatistics}, where the nested variogram model is a composition of $L$ basic variogram structures, and the regionalized variable is a composition of $L$ independent, standard normal, spatial components operating at different geologic scales:
\begin{align}
    \label{eq:gamma_comp}
    \gamma(\mathbf{h}) & = \sum_{\ell=1}^{L} \gamma_{\ell}(\mathbf{h})                     \\
    \label{eq:z_comp}
    Z(\mathbf{u})      & = \sum_{\ell=1}^{L}b_{\ell}Y_{\ell}(\mathbf{u}) + \mu(\mathbf{u})
\end{align}

Where the coefficient $b_{\ell}$ is the square root of the variance contribution of $\gamma_{\ell}$. The key difference between the \gls{NMR} and factorial kriging approaches, is that the regionalized \gls{NMR} variable is a weighted, non-linear combination of independent spatial component as Equation \ref{eq:fpass1} highlights. With the \gls{LMC} or factorial kriging, the $Y_{\ell}$ independent factors are synthetic features of the model and are not directly observed. The \gls{NMR} requires the explicit definition of a pool of independent Gaussian factors. The pool may contain any number of structures that define the network's input dimension. It is simple to generate realizations of the independent Gaussian factors with any unconditional simulation algorithm such as \gls{SGS} \citep{gomez-hernandez1993joint} or LU simulation \citep{davis1987production}.

The choice of covariance structures for the Gaussian pool must consider the end goal of the spatial mixture. The latent factors must be reasonable in the sense that it is possible to achieve the objective. For example, the pool must contain short-range features if the final goal is short-range features. Long-range and short-range structures can produce a final mixture with medium-range features; however, a pool with only long-range structures cannot. Pool considerations include (1) the conceptual geological model, (2) the $L$ nested components of normal score variogram, (3) the $L$ nested components of each indicator variogram model and their potential asymmetry, (4) the down hole connectivity measures from the observed data and the potential connectivity of extreme values, and (5) the composition of the objective function, discussed in more detail in Section \ref{sec:04paraminfer}. The factorial kriging concept provides a reasonable starting point for the design of the Gaussian pool. An initial pool can be inferred through the decomposition of all variogram models into their basic components (as in Equation \ref{eq:gamma_comp}). Each basic variogram component is a single licit structure with three orientation parameters, three range parameters, a nugget of zero and sill of one. By convention a pure nugget latent factor is added as the $0^{th}$ factor.

Consider a small \gls{2D} synthetic example where the goal of the \gls{NMR} is to generate gridded realizations with strongly asymmetric 0.1 and 0.9 quantile indicator variograms. That is, the low-grade two-point spatial continuity differs drastically from the high-grade. This scenario is challenging for multivariate Gaussian simulation algorithms and has been discussed at length by many practitioners \citep{journel1983nonparametric, gomez-hernandez1998be, journel1993entropy,renard2011conditioning, guthke2013non}. The maximum entropy characteristic of the multivariate Gaussian distribution tends towards disconnected extremes and maximum connectivity of intermediate values. The destructuring of indicators away from the median is symmetric. The \gls{NMR} framework is able to overcome this challenge through the use of a well-designed Gaussian pool and $\omega$ bounds. Factor 1 is a highly anisotropic factor, oriented north-south direction and Factor two is isotropic with a range off $\approx \frac{1}{2}$ the domain size. The factors are activated using the \gls{MPL} with $\omega_{1}=4$ emphasizing the high values and $\omega_{2}=1/\omega_{1}$ emphasizing the low values, followed by linear combination (Equation \ref{eq:fpass1}) and normal score transform. The top row of Figure \ref{fig:pool_asymmetry} shows the factors in activated units (left and center), and the normal score transform of the linear mixture (right). The mixture model is univariate Gaussian but not multivariate Gaussian. Note the difference in activation units. The bottom row shows the 0.1 (black) and 0.9 (red) indicator variograms of the final \gls{NMR} mixture in the north-south and east-west directions. The longer range, more-isotropic, low-grade continuity is preserved through factor two and $\omega < 1$, and the highly-anisotropic, higher-grade continuity is preserved through factor one and $\omega > 1$. This small example is illustrative of three key concepts with respect to latent factor design:
\begin{enumerate}[noitemsep]
    \item Latent factors should be designed in conjunction with $\omega$ bounds, targeting specific ranges of continuity, and the final model goals.
    \item The ranges of factors are reduced through mixing. Mixing of factors cannot increase continuity beyond the longest range structures in the pool. Factor two in Figure \ref{fig:pool_asymmetry} has an isotropic variogram range of 64 meters. The range of the 0.1 quantile indicator variogram is reduced to roughly 40 and 25 meters in the north-south and east-west directions, respectively.
    \item Anisotropy ratios of factors are affected through mixing. The anisotropy ratio of factor one is roughly unchanged in Figure \ref{fig:pool_asymmetry} (4:1), however the anisotropy ratio of factor two is increased from 1:1 to $\approx$ 1.6:1. In general, isotropic factors will tend to become more anisotropic, and anisotropic factors will become more isotropic.
\end{enumerate}

\begin{figure}[!htb]
    \begin{subfigure}{1.0\textwidth}
        \centering
        \includegraphics[scale=0.60, max size={\textwidth}{\textheight}]{./0-Figures/04-Ch4/f1_f2_grid.png}
    \end{subfigure}
    \begin{subfigure}{1.0\textwidth}
        \centering
        \includegraphics[scale=0.60, max size={\textwidth}{\textheight}]{./0-Figures/04-Ch4/asymmetric_ind_vario.png}
    \end{subfigure}
    \caption{The top row shows factor one activated with $\omega_{1}=4$ (left), factor two activated with $\omega_{2}=1/\omega_{1}$ (center), and the normal score transform the mixture (right). The bottom row shows the gridded indicator variograms for the 0.1 quantile (black) and 0.9 quantile (red) in the north-south (left) and east-west (right) directions. }
    \label{fig:pool_asymmetry}
\end{figure}

Overall, the Gaussian pool must be reasonable with respect to the optimization targets, discussed in Section \ref{sec:04paraminfer}. The inferred mapping function is non-unique, so there is no ``correct'' Gaussian pool; however, some latent factor combinations may be unreasonable. For example, one cannot expect a Gaussian pool composed entirely of long-range structures to generate short-range features in the \gls{NMR} model, and vice-versa. Similar is true for factor orientations. If certain spatial features are required in certain orientations, factors should be designed to address this. Factors aimed at generating high-order connectivity features should consider the major orientation of drilling. Factors with different orientations may be combined, for example, a north-south and an east-west factor can generate north-east striking features in the final model. This base-case requirement of ``reasonableness'' motivates variogram decomposition as the initial approach to latent factor design. There is no definitive recipe for latent factor design though the following rules of thumb apply:
\begin{enumerate}[noitemsep]
    \item Decomposition of nested variogram models into their basic components   is a reasonable starting point.
    \item Checking feature reproduction with a traditional simulation algorithm (such as \gls{SGS}) can provide insight into the ``missing'' features to be added.
    \item Add factors based on the conceptual geologic model and end goals of the \gls{NMR} model.
    \item Prune any redundant factors. That is, the pool is only as complex as required.
    \item Anisotropy of factors is reduced through mixing. That is, the \gls{NMR} model is only as anisotropic as the most anisotropic factor.
    \item The range of factors is reduced through mixing. That is, the range of \gls{NMR} model is only as long as the longest range factor.
\end{enumerate}

Once a reasonable Gaussian pool is established, the objective function components can be initialized and the parameter vector $\theta$ can be optimized. The following section discusses details of objective function, optimization algorithm and \gls{NMR} parameter inference.


% \begin{enumerate}[noitemsep]
%     \item Conceptual geologic model
%     \item Indicator asymmetry
%     \item Connectivity of extreme values
%     \item nugget
% \end{enumerate}


\FloatBarrier
\section{Parameter Inference}
\label{sec:04paraminfer}

The \gls{NMR} approximates the mapping function, $\mathcal{F}_{\theta}$, between latent and observed space. This function is parameterized by the unknown vector $\theta$. The unknown parameters are inferred from known features: (1) the observed data, (2) the specified latent pool, and (3) the objective function components. Features (2) and (3) are specified by the user and guide the parameter optimization process. $\theta$ is a vector of $2 \cdot (M+1)$ real values of factor weights and \gls{MPL} exponents:
\begin{equation}
    \theta = \{ a_{0}, \dots, a_{M}, \omega_{0}, \dots, \omega_{M} \}
    \label{eq:theta}
\end{equation}

and is optimized with the heuristic, genetic algorithm \gls{DE} \citep{price2013differential}. A population of candidate solutions is initialized and then evolved in a process mimicking natural selection. Each member of the population has an associated ``fitness'' value, and the fittest members of the population are carried over to subsequent generations. Through multiple generations of mutation and crossover operations, the candidate solutions converge towards a solution that minimizes the objective function. Parameter inference begins by simulating a set of $L$ x $M+1$ unconditional realizations at the input data locations, where $L$ is the number of realizations and $M+1$ is the number of independent factors, including the nugget. These unconditional realizations permit the evaluation of the objective function as $\theta$ is evolved.

\subsection{Differential Evolution}
\label{subsec:04de}

Optimization of network weights uses a gradient-free, heuristic genetic algorithm. Gradient-free methods are typically employed when information about the derivative of the objective function is either costly to obtain or unreliable and noisy \citep{conn2009introduction}. \Gls{DE} is a global stochastic search algorithm that is practical for non-linear, non-differentiable objective functions with a necessarily large search space \citep{rios2013derivativefree}. Any objective function is permissible, which is an advantage of \gls{DE}.

\Gls{DE} is based on natural evolutionary processes where the fittest member of the population survives through a ``natural selection'' process. An initial population of size $NP$ x $D$ is generated by randomly sampling the objective function space within the defined constraints. $NP$ is the number of individuals in the population, and $D$ is the dimensionality of the problem. Each vector from the initial population is passed through the objective function to evaluate its evolutionary ``fitness''. A mutant vector is then generated from the population by adding the scaled difference between two randomly selected vectors to a third randomly selected vector \citep{price2013differential}. A trial vector is generated by recombining the mutant vector with the initial population's current row vector considering a user-defined crossover probability. The trial vector's fitness is compared to the current population vector's fitness in an evolutionary sense. If the trial vector's fitness exceeds the current population vector's, it replaces the current population vector. Each iteration of the algorithm compares all population vectors to a randomly generated trial vector and accepts the trial vector if its fitness is greater than the current vector. The ``surviving'' vectors from each algorithm iteration become the parent vectors or the next iteration population.

\Gls{DE} minimizes the objective function after simulation and scaling of the objective components. A population of candidate vectors (Equation \ref{eq:theta}) is randomly initialized, and each member's fitness is evaluated. A population size of 30-50 is a reasonable balance between sufficient diversity and algorithm runtime though there is no definitive guideline for population size \citep{piotrowski2017review}. Over a specified number of iterations, each member of the population is mutated with other members. Optimization employs a DE/current-to-best/1 mutation strategy. The mutated vector is described by \citep{georgioudakis2020comparative}:
\begin{equation}
    \mathbf{v}_{i} = \mathbf{x}_{i} + F(\mathbf{x}_{best}-\mathbf{x}_{i}) + F(\mathbf{x}_{r1} - \mathbf{x}_{r2})
    \label{eq:mutation}
\end{equation}

Where $r1 \neq r2 \neq i$, $F$ is the scaling factor controlling the amplification of the difference vectors and $\mathbf{x}_{best}$ is the current best vector in the population. The mutation process learns from the current best vector (local searching of the solution space) while exploring the global search space through the randomly selected difference vector. The parameter $F$ is analogous to the learning rate in machine learning problems. A smaller $F$ value will lead to smaller mutation step sizes, and the algorithm will take longer to converge. Larger $F$ values increase the degree of solution exploration but may lead to divergence. \cite{price2013differential} suggests there is no upper limit for $F$, however effective values are almost always $F < 1.0$.

This new mutated ``genetic information'' is crossed over to other population members based on a crossover probability, $CR$ and a binomial crossover scheme. The trial vector is built from two vectors: the mutant and another member of the population. The crossover is described by \citep{price2013differential}:
\begin{equation}
    \mathbf{t}_{i} =
    \begin{cases}
        \mathbf{v}_{i, j}, & \text{ if }rand_{j}(0,1) \leq CR \text{ or } j=j_{rand} \\
        \mathbf{x}_{i, j}, & \text{ otherwise }
    \end{cases}
    \label{eq:crossover}
\end{equation}

Where $\mathbf{t}_{i}$ is the trial vector, $\mathbf{x}_{i, j}$ is the target vector, $j$ denotes the index of the $i^{th}$ member of the population, $rand_{j}(0,1)$ is a uniform random number $\in [0,1]$, and $j_{rand}$ is a random index. For each element in the target vector, if the uniform random number is less than $CR$, the element is copied from the mutant vector $\mathbf{v}_{i, j}$; otherwise, the target element remains. The random index $j_{rand}$ ensures the target vector is not copied completely; at least one element of the mutant vector passes to the trial vector. $CR$ influences the diversity of the evolving population. Larger $CR$ values introduce more variation in the population, resulting in a greater search of the global solution space, while smaller values may lead to stagnation \citep{georgioudakis2020comparative}. Optimization employs a non-linear crossover scheme following the work of \cite{mohamed2014rdel}:
\begin{equation}
    CR = CR_{hi} + (CR_{lo} - CR_{hi}) \cdot (1-i/N)^{4}, \ \ i=1,\dots, N
    \label{eq:f(cross)}
\end{equation}

The idea behind the non-linear crossover scheme is that when population variance is high in early generations, the crossover rate is low. This crossover scheme prevents extreme diversity or potential divergence in the initial iterations. In subsequent generations, the variance of the population decreases as the vectors become similar, approaching the solution. In order to thoroughly explore this more local search space, the crossover should be high, encouraging a diverse population of ``good'' solutions.

\subsection{Objective Function}
\label{subsec:04objfunc}

The parameter vector $\theta$ is optimized heuristically through an objective function and \gls{DE}. This approach is highly flexible as the objective function can contain any number of components. The practitioner provides target features of the final mixture from which a loss or objective value can be calculated for the candidate mixture. The objective value is minimized through successive iterations reproducing the target spatial features. The possible objective function components are: (1) the continuous variogram, (2) indicator variograms, (3) cumulative run length frequencies, and (4) the $n$-point connectivity function. Components may have multiple sub-components for each indicator threshold. Run length frequencies and the $n$-point connectivity function are higher-order multi-point statistics that better characterize non-Gaussian features than the two-point variogram. The objective function comprises a weighted combination of $C$ objective components:
\begin{equation}
    O = \sum_{c=1}^{C} w_{c}O_{c}
    \label{eq:fobj}
\end{equation}

Where $w_{c}$ is the component weight and $O_{c}$ is component objective value. The objective function quantifies how different the desired feature is from the target feature. Weighting the objective function components is required as they may be represented in widely different units. For example a variogram component is expressed in units of variance squared ($\sum [ \gamma_{\text{target}} - \gamma_{\text{realization}}]^{2}$) while cumulative runs is expressed as the number of runs squared ($\sum [ R_{\text{target}} - R_{\text{realization}}]^{2}$), and may be orders of magnitude different. A weighting scheme follows the work of \cite{deutsch1992annealing}. The goal is to have each component contribute equally to the overall objective value where each weight is inversely proportional to the average change of that objective component \citep{deutsch1992annealing}:
\begin{equation}
    w_{c} = \frac{1}{\bar{|\Delta O_{c}|}}
    \label{eq:fobj_wt}
\end{equation}

The average change of the objective component is approximated numerically by averaging the change of $J=1000$ independent forward passes through the network:
\begin{equation}
    \bar{|\Delta O_{c}|} = \frac{1}{J} \sum_{j=1}^{J} | O^{j}_{c} - O_{c}|, \ \ c=1,\dots, C
    \label{eq:fobj_avg}
\end{equation}

Where $\bar{|\Delta O_{c}|}$ is the average change of component $c$, $O^{j}_{c}$ is the updated objective value for iteration $j$ and $O_{c}$ is the initial objective value of that component. Though any number of components can enter the objective function, one should consider contrasting objectives. For example, one cannot expect to closely reproduce continuous and indicator variograms if the principal directions of continuity are orthogonal. It is less clear how the specification of run frequencies or connectivity functions affect variograms, though there are likely confounding factors. The objective value is the \gls{MSE} between the experimental values calculated on the distribution derived from Equation \ref{eq:fpass1} and the initialized target values. The objective function (Equation \ref{eq:fobj}) is evaluated across all simulated realizations and minimized in expectation.

\subsection{Precedence}
\label{subsec:04precedence}

Application of precedence is an optional component of parameter inference. Precedence allows the spatial features of certain latent factors to be prioritized in the mapping function. A particular factor can be given precedence during imputation if one desires its spatial features in certain portions of the grade range. Precedence is achieved by employing a sigmoid weighting function where $s(y) = \frac{1}{1+e^{-y}}$. The weighting function modifies the forward pass (Equation \ref{eq:fpass1}) through the network to:
\begin{align}
    \label{eq:wtpass1}
    \mathbf{x} & = a_{p} \cdot \phi(\mathbf{y}_{p}) + \sum_{m=0}^{M-1} \phi(\mathbf{y}_{m}) \cdot s(\mathbf{y}_{p} \cdot x), \ \ m \neq p \\
    \label{eq:wtpass2}
    \mathbf{z} & = G^{-1}\left( F_{X}\left(\mathbf{x} \right)\right)
\end{align}

Where $a$ is the weight to the factor, subscript $p$ is the factor index with precedence, $\phi(\dots)$ is the power law activation function, $s(\dots)$ is the sigmoid function, and $x$ is a constant $\in [-10, 10]$. The sign and magnitude of $x$ controls what part of the grade range factor $\mathbf{y}_{p}$ influences. If $x<0$ $\mathbf{y}_{p}$ influences high values and $x>0$ influences low values. Figure \ref{fig:sigmoid} shows a weighting function where $x=-1.5$. The y-axis is the weight to the remaining factors for the range of values of $\mathbf{y}_{p}$. In this scenario $\mathbf{y}_{p}$ receives $\approx 95\%$ of the weight when equal to 2.0, $\approx 80\%$ of the weight when equal to 1.0 and so on. As the magnitude of $x$ increases, $s(\dots)$ tends towards a binary step function centered on zero.

\begin{figure}[htb!]
    \centering
    \includegraphics[scale=0.6, max size={\textwidth}{\textheight}]{./0-Figures/04-Ch4/sigmoid.png}
    \caption{Sigmoid weighting function where $x=-1.5$. The x-axis is the range of $\mathbf{y}_{p}$ and the y-axis is the weight to the remaining M-1 factors. }
    \label{fig:sigmoid}
\end{figure}

The sigmoid weighting function allows a certain factor to take precedence where the observed data takes on either high or low values. This increased control is particularly useful for capturing local high- or low-grade continuity with well-designed latent factors.

\subsection{Complete Algorithm}
\label{subsec:04algorithm}

Algorithm \ref{alg:nmropt} summarizes all components of the \gls{NMR} parameter inference workflow. Unconditional simulation of the latent Gaussian pool at the data locations is the first step (lines 2-6). The objective function components are scaled using the unconditional realizations (lines 7-10) prior to heuristically optimizing $\theta$ with \gls{DE} (lines 11-35).

\begin{algorithm}
    \caption{\gls{NMR} parameter inference pseudocode.}\label{alg:nmropt}
    \begin{algorithmic}[1]
        \State simulate unconditional realizations (LU or SGS):
        \For{$\ell = 1, \dots, L$}
        \For{$m = 1, \dots, M$}
        \State simulate $\mathbf{Y^{\ell}_m}$
        \EndFor
        \EndFor
        \State scale objective components:
        \For{$c = 1, \dots, C$}
        \State $w_{c} = \frac{1}{\bar{|\Delta O_{c}|}}$
        \EndFor
        \State initialize population: $pop(pool dim, popsize)$ \Comment{candidate solutions}
        \State calculate fitness of population: $fpop$
        \State $best = minloc(fpop)$ \Comment{index of initial best solution}
        \For{i=1,\dots, its} \Comment{begin DE}
        \For{j=1,\dots, popsize}
        \State $mutant = \Call{mutation}{pop(j)}$ \Comment{best-to-current mutation}
        \State $\theta = \Call{crossover}{mutant}$ \Comment{binomial crossover}
        \State $a, \omega = \Call{vector\_to\_matrices}{\theta}$ \Comment{reshape vector to $a, \omega$ vectors}
        \State $O = 0.0$
        \For{$\ell = 1, \dots, L$}
        \State $z = \Call{network\_forward}{a, \omega}$ \Comment{forward pass through network}
        \State $iz = \Call{indicator\_transform}{z, thresholds}$
        \For{$c = 1, \dots, C$}
        \State $O_{c} = \sum [ \gamma^{\text{target}}_{c} - \gamma^{\text{realization}}_{c}]^{2}$
        \EndFor
        \State $O = \sum_{c=1}^{C}w_{c}\cdot O_{c}$
        \EndFor
        \State $O = O / L$
        \If{$O < fpop(best)$}
        \State $fpop(best) = O$ \Comment{update objective value}
        \State $pop(best) = \theta$ \Comment{retain the trial}
        \State $best = j$ \Comment{track the new best}
        \EndIf
        \EndFor
        \EndFor \Comment{end DE}
    \end{algorithmic}
\end{algorithm}

Inference of $\theta$ is relatively straightforward with only a handful of \gls{DE} parameters, though care must be given to align the design of the latent pool and objective components with the final goals of the \gls{NMR} model. Depending on the modeling scenario, this is the most challenging portion of the workflow.

\FloatBarrier
\subsection{Checks}
\label{subsec:04checks}

To highlight the concepts of parameter checking a small \gls{3D} synthetic example with 746 data is generated. The data set is simulated such that there is asymmetric destructuring of the indicator variograms away from the median and connectivity of high-grade values. These features provide a reasonable test case of the application of the \gls{NMR} workflow. Imputation concepts are presented using the same data set in Chapter \ref{ch:05impute}. Figure \ref{fig:synthetic_sections} shows a plan-view and long-section through the synthetic data in Gaussian units. Figure \ref{fig:synthetic_varios} shows experimental variogram points and fitted models for the normal score variable (a), the 0.1 indicator (b) and 0.9 indicator (c). For brevity the median indicator variogram is not shown however the indicator variograms show asymmetric destructuring about the median. Figure \ref{fig:synthetic_npoint} shows the $n$-point connectivity function which approaches zero at six connected steps.

\begin{figure}[htb!]
    \centering
    \includegraphics[scale=0.6, max size={\textwidth}{\textheight}]{./0-Figures/04-Ch4/synthetic_sections.png}
    \caption{Plan-view (left) and long-section oriented $340\degree$ (right) though the synthetic data set. Both sections have a tolerance pf $\pm25$ meters.}
    \label{fig:synthetic_sections}
\end{figure}

\begin{figure}
    \begin{subfigure}{1.0\textwidth}
        \centering
        \includegraphics[scale=0.6, max size={\textwidth}{\textheight}]{./0-Figures/04-Ch4/syn_cont_vario.png}
        \caption{Normal Scores}
    \end{subfigure}
    \begin{subfigure}{1.0\textwidth}
        \centering
        \includegraphics[scale=0.6, max size={\textwidth}{\textheight}]{./0-Figures/04-Ch4/syn_ivario_0_1.png}
        \caption{0.1 Quantile}
    \end{subfigure}
    \begin{subfigure}{1.0\textwidth}
        \centering
        \includegraphics[scale=0.6, max size={\textwidth}{\textheight}]{./0-Figures/04-Ch4/syn_ivario_0_9.png}
        \caption{0.9 Quantile}
    \end{subfigure}
    \caption{Fitted experimental variograms for the normal score variable (a), the 0.1 indicator (b) and 0.9 indicator (c). Note the asymmetric destructuring of the indicator variograms. }
    \label{fig:synthetic_varios}
\end{figure}

\begin{figure}[htb!]
    \centering
    \includegraphics[scale=0.6, max size={\textwidth}{\textheight}]{./0-Figures/04-Ch4/synthetic_npoint.png}
    \caption{$N$-point connectivity function above the 0.9 quantile indicator for the synthetic data set.}
    \label{fig:synthetic_npoint}
\end{figure}

The latent factor pool contains two structures: the long-range structure from the 0.1 quantile indicator model and the single structure from the 0.9 quantile indicator model. The goal of the pool is to have the long-range structure spatially control the lower grades and have the short range structure spatially control the highs. The omega parameters are constrained such that $0.25 < \omega_{1} \leq 1.0$ and $2.0 < \omega_{2} \leq 4.0$.

The primary \gls{NMR} parameter check is reproduction of objective function components. The mapped unconditional realizations, $\mathcal{F}_{\theta}\left(\mathbf{Y}\right)$, must reproduce, on average, the input objective targets. The objective function components are the normal score variogram model, the 0.1 indicator variogram model and the 0.9 indicator variograms (Figure \ref{fig:synthetic_varios} (a), (b), and (c)), and the $n$-point connectivity function (Figure \ref{fig:synthetic_npoint}).




\begin{enumerate}[noitemsep]
    \item Introduce 3D example here?
    \item Differential evolution with small example, PDE
    \item Objective function
    \item Constraints and precedence
    \item Sigmoid weighting
    \item Conflicting objectives
    \item Checks
\end{enumerate}


\FloatBarrier
\section{Implementation Details}
\label{sec:04implementd}

\begin{enumerate}[noitemsep]
    \item Number of data
    \item Number of uncond. reals
    \item Variogram approximation
    \item Conflicting objectives
    \item Non-uniqueness
\end{enumerate}


\FloatBarrier
\section{Discussion}
\label{sec:04discuss}