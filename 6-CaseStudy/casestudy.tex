%!TEX root = ../Thesis_example_compile_this.tex

%%%%%%%%%%%%%%%%%%%%%%%%%%%%%%%%%%%%%%%%%%%%%%%%%%%%%%%%%%%%%%%%%%%%%%%%%%%%%%%
\chapter{Case Study}
\label{ch:casestudy}
%%%%%%%%%%%%%%%%%%%%%%%%%%%%%%%%%%%%%%%%%%%%%%%%%%%%%%%%%%%%%%%%%%%%%%%%%%%%%%%

This chapter presents the application of the complete \gls{NMR} workflow to a real \gls{3D} dataset. The \gls{NMR} workflow is designed to capture the spatial features of non‐Gaussian distributions in a way that the multi-Gaussian assumption and a \gls{LMR} cannot. The goal of the case study is to construct an \gls{NMR} spatial model and compare it to a traditional model. The dataset comes from a mineral deposit where the project operator have noted non-Gaussian characteristics within drillholes.

The workflow consists of (1) inferring the parameters that map the latent space to the observed space with \texttt{NMROPT} and (2) imputing the latent factors at the data locations with \texttt{NMRIMP} that return the true values when mapped. The mapping is non-linear and inferred such that all desired two- and multipoint statistics are captured in the final realizations. The idea is that the \gls{NMR} realizations can capture non-Gaussian spatial features such as asymmetric indicator variograms and connectivity of extreme values. Twenty-five conditional realizations are generated with the \gls{NMR}, and pertinent model statistics are checked and compared to \gls{SGS}.

\FloatBarrier
\section{Exploratory Data Analysis}
\label{sec:eda}

The complete dataset consists of $27,983$ values nominally composited to 5 meters. All coordinates are transformed such that the spatial centroid of the data is $(x=0,y=0,z=0)$. For cross-validation purposes, approximately 30\% of the data (predominantly drilled in higher-grade areas) is withheld as a `test set'. The test data is not exposed to any subsequent modeling step. Figure \ref{fig:datasets} shows \glspl{CDF} and basic summary statistics for the `training' and `test' datasets. All subsequent modeling considers only the training dataset and the test set is utilized in Section \ref{sec:compare}.

\begin{figure}[htb!]
    \centering
    \includegraphics[scale=0.75, max size={\textwidth}{\textheight}]{./0-Figures/06-Ch6/test_train_cdf.png}
    \caption{ \Glspl{CDF} for training and testing datasets. }
    \label{fig:datasets}
\end{figure}

Experimental variograms are first calculated using an omni-directional search that informs the nugget effect. This search uses short lags to highlight shape and range of the short scale continuity. Figure \ref{fig:omni} suggests a nugget effect of near zero, and that an exponential variogram shape is appropriate for the modeled variable. Directional experimental variograms are calculated first in original units and fitted with an exponential model. Figure \ref{fig:orig_expvar} shows the fitted experimental points and Table \ref{tab:orig_expvar} shows the model parameters. The original units variogram defines the search orientation and anisotropy for estimator based declustering. Indicator variograms are calculated for the 0.1, 0.5 and 0.9 quantiles of the grade distribution. Figures \ref{fig:ind_expvar} shows the fitted indicator variogram models and Table \ref{tab:ind_expvar} summarizes the model parameters.

% \begin{table}[!htb]
%     \centering
%     \caption{Training and testing datasets.}
%     \resizebox{1\width}{!}{\begin{tabular}{ccc}
\toprule
{} &  VAR\_A TRAIN &  VAR\_A TEST \\
\midrule
\textbf{count} &     21582.00 &     6455.00 \\
\textbf{mean } &         1.13 &        1.29 \\
\textbf{std  } &         1.22 &        1.14 \\
\textbf{min  } &         0.00 &        0.00 \\
\textbf{25\%  } &         0.07 &        0.22 \\
\textbf{50\%  } &         0.82 &        1.13 \\
\textbf{75\%  } &         1.77 &        1.97 \\
\textbf{max  } &         9.09 &        8.88 \\
\bottomrule
\end{tabular}
}
%     \label{tab:datasets}
% \end{table}


\foreach \sec/\name in \sectuples
{
    \begin{figure}[htb!]
        \centering
        \includegraphics[scale=0.5, max size={\textwidth}{\textheight}]{./0-Figures/06-Ch6/var_a_\sec.png}
        \caption{ \name \ sections through the training dataset. }
        \label{fig:train_\sec}
    \end{figure}
}

\begin{figure}[htb!]
    \centering
    \includegraphics[scale=0.75, max size={\textwidth}{\textheight}]{./0-Figures/06-Ch6/var_a_omni.png}
    \caption{Omni-directional variogram for nugget inference. }
    \label{fig:omni}
\end{figure}

\begin{figure}[htb!]
    \centering
    \includegraphics[scale=0.6, max size={\textwidth}{\textheight}]{./0-Figures/06-Ch6/var_a_variogram.png}
    \caption{Fitted original unit variograms in the major, minor and tertiary directions. }
    \label{fig:orig_expvar}
\end{figure}

\begin{table}[!htb]
    \centering
    \caption{Original unit variogram model parameters.}
    \resizebox{1\width}{!}{\begin{tabular}{cccc}
\toprule
{} & Nugget &  Structure 1 &  Structure 2 \\
\midrule
\textbf{Contribution} &  0.000 &        0.655 &        0.345 \\
\textbf{Model Shape } &        &  exponential &  exponential \\
\textbf{Angle 1     } &        &        277.0 &        277.0 \\
\textbf{Angle 2     } &        &         20.0 &         20.0 \\
\textbf{Angle 3     } &        &         30.0 &         30.0 \\
\textbf{Range 1     } &        &         36.3 &        250.0 \\
\textbf{Range 2     } &        &         33.2 &        125.0 \\
\textbf{Range 3     } &        &         33.2 &        100.0 \\
\bottomrule
\end{tabular}
}
    \label{tab:orig_expvar}
\end{table}

\begin{figure}
    \begin{subfigure}{1.0\textwidth}
        \centering
        \includegraphics[scale=0.6, max size={\textwidth}{\textheight}]{./0-Figures/06-Ch6/var_a_ivariogram_0_1.png}
        \caption{0.1 Quantile}
    \end{subfigure}
    \begin{subfigure}{1.0\textwidth}
        \centering
        \includegraphics[scale=0.6, max size={\textwidth}{\textheight}]{./0-Figures/06-Ch6/var_a_ivariogram_0_5.png}
        \caption{0.5 Quantile}
    \end{subfigure}
    \begin{subfigure}{1.0\textwidth}
        \centering
        \includegraphics[scale=0.6, max size={\textwidth}{\textheight}]{./0-Figures/06-Ch6/var_a_ivariogram_0_9.png}
        \caption{0.9 Quantile}
    \end{subfigure}
    \caption{Fitted experimental indicator variograms for the 0.1 (a), 0.5 (b) and 0.9 (c) quantiles.}
    \label{fig:ind_expvar}
\end{figure}

\begin{table}[!htb]
    \centering
    \caption{Indicator variogram model parameters. All models have zero nugget.}
    \resizebox{0.9\width}{!}{\begin{tabular}{ccccccc}
\toprule
{} & \multicolumn{2}{l}{0.1 Quantile} & \multicolumn{2}{l}{0.5 Quantile} & \multicolumn{2}{l}{0.9 Quantile} \\
{} &  Structure 1 &  Structure 2 &  Structure 1 &  Structure 2 &  Structure 1 &  Structure 2 \\
\midrule
\textbf{Contribution} &        0.530 &        0.470 &        0.489 &        0.511 &        0.359 &        0.641 \\
\textbf{Model Shape } &  exponential &  exponential &  exponential &  exponential &  exponential &  exponential \\
\textbf{Angle 1     } &        277.0 &        277.0 &        277.0 &        277.0 &        277.0 &        277.0 \\
\textbf{Angle 2     } &         20.0 &         20.0 &         20.0 &         20.0 &         20.0 &         20.0 \\
\textbf{Angle 3     } &         30.0 &         30.0 &         30.0 &         30.0 &         30.0 &         30.0 \\
\textbf{Range 1     } &         13.2 &        200.0 &         18.0 &        150.0 &          8.5 &         61.8 \\
\textbf{Range 2     } &         11.0 &         50.3 &         17.4 &        121.0 &          5.1 &         56.9 \\
\textbf{Range 3     } &         19.3 &         91.3 &         32.5 &         84.5 &          5.1 &         53.3 \\
\bottomrule
\end{tabular}
}
    \label{tab:ind_expvar}
\end{table}

Transformation to Gaussian space requires a representative \gls{CDF}. Declustering weights are produced using \gls{ID2} estimation. A grid of 5x5x5 meter blocks is produced considering a 25-meter buffer in the x, y, and z directions around all available data (the \gls{SMU} grid). \Gls{ID2} estimation weights for each data are accumulated during an estimation run and used as declustering weights. The idea is that data located in dense regions receive less total weight as many data are present in the search neighbourhood. In sparse areas data receive more weight as there are fewer data in the search neighbourhood. Estimator declustering is more robust than traditional cell declustering in the presence of irregular data configurations as the principal orientation of continuity can be captured by the search ellipse. The limits of the estimation grid must be well-defined to prevent unintended edge effects \cite{Wilde2007}. Figure \ref{fig:declus_cdf} shows the clustered (black) and declustered \gls{CDF} (red).

After normal score transform the experimental variogram of the normal scores is calculated and fitted with a model. Figure \ref{fig:ns_expvar} shows the fitted experimental points and Table \ref{tab:ns_expvar} shows the model parameters. A normal score variogram permits calculation of accuracy metrics, particularly the actual fraction of true values that fall within specified probability intervals. Figure \ref{fig:ns_xval} shows a leave-one-drillhole out cross validation and accuracy plot for the fitted normal score variogram model. The scatter plot shows a reasonable \gls{SOR} and reproduction of the true mean. The accuracy plot suggests a reasonable space of uncertainty for the fitted model. It is interesting to note the 0.1 and 0.9 indicator variograms are not highly asymmetric with this dataset.

\begin{figure}[htb!]
    \centering
    \includegraphics[scale=0.75, max size={\textwidth}{\textheight}]{./0-Figures/06-Ch6/var_a_declus_cdf.png}
    \caption{ Declustered \gls{CDF} considering \gls{ID2} weights. }
    \label{fig:declus_cdf}
\end{figure}

\begin{figure}[htb!]
    \centering
    \includegraphics[scale=0.6, max size={\textwidth}{\textheight}]{./0-Figures/06-Ch6/ns_var_a_variogram.png}
    \caption{Fitted normal score variograms in the major, minor and tertiary directions. }
    \label{fig:ns_expvar}
\end{figure}

\begin{table}[!htb]
    \centering
    \caption{Normal score variogram model parameters.}
    \resizebox{1\width}{!}{\begin{tabular}{cccc}
\toprule
{} & Nugget &  Structure 1 &  Structure 2 \\
\midrule
\textbf{Contribution} &  0.000 &        0.668 &        0.332 \\
\textbf{Model Shape } &        &  exponential &  exponential \\
\textbf{Angle 1     } &        &        277.0 &        277.0 \\
\textbf{Angle 2     } &        &         20.0 &         20.0 \\
\textbf{Angle 3     } &        &         30.0 &         30.0 \\
\textbf{Range 1     } &        &         47.6 &        300.0 \\
\textbf{Range 2     } &        &         45.5 &        124.9 \\
\textbf{Range 3     } &        &         45.5 &        100.0 \\
\bottomrule
\end{tabular}
}
    \label{tab:ns_expvar}
\end{table}

\begin{figure}[htb!]
    \centering
    \includegraphics[scale=0.75, max size={\textwidth}{\textheight}]{./0-Figures/06-Ch6/nsvario_xval.png}
    \caption{Normal score variogram \gls{LOOCV} (left) and accuracy plot (right).}
    \label{fig:ns_xval}
\end{figure}

The final step of \gls{EDA} is to generate a reference model for validation purposes. An exhaustive model is generated with \gls{OK} considering a minimum of 6 and a maximum of 40 composites using a search ellipse with ranges and angles aligned with the fitted original units variogram model (Table \ref{tab:orig_expvar}).

\FloatBarrier
\section{Network Parameter Inference}
\label{sec:param}

Parameterization of the \gls{NMR} requires the definition of (1) optimization targets, (2) a pool of latent Gaussian factors, (3) architecture of the network and (4) optimization hyperparameters. The continuous normal score, and indicator variograms from Section \ref{sec:eda} become optimization targets for parameterizing the \gls{NMR}. Target sequence and runs distributions are calculated internally from the drillhole data by \texttt{NMROPT}.

The initial step in defining the Gaussian pool is checking how well a traditional simulation algortithm (e.g.\ \gls{SGS}) reproduces the optimization target only considering the continuous variogram. This ``base-case'' model provides insight into what features are required to achieve the desired targets. \Gls{SGS} performs well reproducing the continuous and all indicator variograms; this is likely due to the data density of the training dataset. Though variograms are well reproduced, there is a discrepancy between downhole sequences from the data and the base-case model. Figure \ref{fig:data_sgs_runs} shows cumulative run length frequencies above the 0.9 quantile calculated directly from the data (red) and the expected value of gridded \gls{SGS} realizations (black). This discrepancy suggests additional high-grade continuity is required up to a length of approximately 30m (six connected five-meter composites).

\begin{figure}[htb!]
    \centering
    \includegraphics[scale=0.75, max size={\textwidth}{\textheight}]{./0-Figures/06-Ch6/data_sgs_runs.png}
    \caption{Cumulative run length frequencies above the 0.9 quantile from the drillhole data and the base-case \gls{SGS} model.}
    \label{fig:data_sgs_runs}
\end{figure}

Decomposition of the target variogram structures into individual components forms the initial latent Gaussian pool. That is, each nested variogram structure defines the covariance of a latent factor with zero nugget and sill of 1.0. As the continuous variogram also reproduces the indicator variograms the initial pool consists of the two nested structures in Table \ref{tab:ns_expvar} for a total of $2 + 1 = 3$ (plus one for the nugget effect) factors. Additional structures are added to pool that correspond with high-grade features of the conceptual geological model and the overall orientation of drilling. As a pure nugget factor is added to the pool, any structure with a major range less than 30 meters is pruned to prevent adding too much noise to the network inputs. Though the network can filter this noise, experiments show some level of pruning improves the final parameter inference. Table \ref{tab:pool} shows the covariance structure of each component in the final pool, excluding the nugget factor. Factors 3 and 4 correspond to high-grade structures.


% Additional structures are added to pool that correspond with high-grade clusters. The grade distribution is thresholded at the $95^{th}$ quantile generating a binary indicator. Agglomerative clustering \citep{mullner2011modern} generates spatially coherent clusters using x, y, z coordinates and the continuous variable. Cluster orientations and anisotropies are determined from the eigenvectors and eigenvalues of the 3x3 covariance matrix of cluster coordinates. The x, y, and z components of the eigenvectors form \gls{GSLIB} style rotation angles as:
% \begin{align}
%     \text{azm} & = 90 - \text{atan2}(y, x) * 180/\pi               \\
%     \text{dip} & = \text{atan2}(z, \sqrt{x^{2} + y^{2}}) * 180/\pi
%     \label{eq:azmdip}
% \end{align}

\begin{table}[!htb]
    \centering
    \caption{Covariance structures of the Gaussian pool (excluding the nugget).}
    \resizebox{0.9\width}{!}{\begin{tabular}{cccc}
\toprule
{} &     Factor 1 &     Factor 2 &   Factor 3 \\
\midrule
\textbf{Contribution} &        1.000 &        1.000 &      1.000 \\
\textbf{Model Shape } &  exponential &  exponential &  spherical \\
\textbf{Angle 1     } &        277.0 &        277.0 &        0.0 \\
\textbf{Angle 2     } &         20.0 &         20.0 &      -90.0 \\
\textbf{Angle 3     } &         30.0 &         30.0 &        0.0 \\
\textbf{Range 1     } &         47.6 &        300.0 &      200.0 \\
\textbf{Range 2     } &         45.5 &        124.9 &      100.0 \\
\textbf{Range 3     } &         45.5 &        100.0 &      100.0 \\
\bottomrule
\end{tabular}
}
    \label{tab:pool}
\end{table}

The choice of activation function exponent, $\omega$, controls each factor's influence on either high or low values. $\omega < 1$ emphasizes low values (<0 in Gaussian units) and $\omega > 1$ emphasizes high values (>0 in Gaussian units). Table \ref{tab:omega} shows the minimum and maximum bound for each trainable $\omega$ parameter. Factors 1 and 2 are constrained to be relatively neutral, that is influencing both high and low values but centered about $1.0$, while Factors 3 and 4 are constrained to influence high values only. The nugget effect factor is constrained to $\omega=1$.

\begin{table}[!htb]
    \centering
    \caption{$\omega$ bounds by factor.}
    \resizebox{0.9\width}{!}{\begin{tabular}{ccccc}
    \toprule
    {}              & $\omega_{min}$ & $\omega_{max}$ & precedence & $\sigma$ wt \\
    \textbf{Factor} &                &                &            &             \\
    \midrule
    \textbf{1     } & 0.25           & 4.00           & 4          & -           \\
    \textbf{2     } & 0.25           & 4.00           & 3          & -           \\
    \textbf{3     } & 1.00           & 4.00           & 1          & -1.50       \\
    \textbf{4     } & 0.25           & 4.00           & 2          & -           \\
    \bottomrule
\end{tabular}
}
    \label{tab:omega}
\end{table}

The network architecture is dictated by the number of factors in the Gaussian pool. The pool consists of 4 structures, plus one for the nugget, for a total of 5 latent Gaussian factors, resulting in an input layer of five nodes. Each input node has a single connection weight to a single hidden layer of the same dimension as the input. The network has a total of $2 \cdot M = 10$ trainable parameters; five connection weights and five $\omega$ values.

Table \ref{tab:de_params} shows \gls{DE} parameters. The indicator thresholds used for optimization correspond to the indicator variograms in Section \ref{sec:eda}, that is, $G^{-1}([0.1, 0.5, 0.9])$ where $G^{-1}$ is the inverse of the Gaussian \gls{CDF}. Runs and $n$-point connectivity are calculated up to a maximum of 10 steps ($\approx$50m).

\begin{table}[!htb]
    \centering
    \caption{Differential Evolution parameters.}
    \resizebox{0.9\width}{!}{\begin{tabular}{cccccccc}
\toprule
{} &  Population &  Lower bound &  Upper bound &    F &  CR lo &  CR hi &  Iterations \\
\midrule
\textbf{value} &          30 &         0.25 &         1.00 & 0.80 &   0.10 &   0.80 &        1500 \\
\bottomrule
\end{tabular}
}
    \label{tab:de_params}
\end{table}

Network optimization is run for 1500 iterations resulting in a reasonably stable objective function value. The objective function value (Figure \ref{fig:fobj}) does not reach zero as it is the expected value across all realizations. Though the objective function components are weighted, any deviation in the number of runs can lead to larger objective function values. Table \ref{tab:opt_omega} shows the final optimized connection weight and  $\omega$ parameters. Figures \ref{fig:nmropt_uncond_real0_xy} to \ref{fig:nmropt_uncond_real0_yz} show the first unconditional realization at the data locations generated by \texttt{NMROPT}. Objective components are calculated using these realizations.

\begin{figure}[htb!]
    \centering
    \includegraphics[scale=0.75, max size={\textwidth}{\textheight}]{./0-Figures/06-Ch6/fobj.png}
    \caption{\Gls{DE} objective value versus iteration.}
    \label{fig:fobj}
\end{figure}

\begin{table}[!htb]
    \centering
    \caption{Optimal connection weight and $\omega$ parameters.}
    \resizebox{0.9\width}{!}{\begin{tabular}{ccc}
\toprule
{} &  Weight &  $\omega$ \\
\textbf{Factor} &         &           \\
\midrule
\textbf{1     } &    0.77 &      3.98 \\
\textbf{2     } &    0.43 &      3.72 \\
\textbf{3     } &    0.34 &      1.00 \\
\textbf{4     } &    0.15 &      1.00 \\
\bottomrule
\end{tabular}
}
    \label{tab:opt_omega}
\end{table}

\foreach \sec/\name in \sectuples
{
    \begin{figure}[htb!]
        \centering
        \includegraphics[scale=0.5, max size={\textwidth}{\textheight}]{./0-Figures/06-Ch6/nmropt_uncond_real0_\sec.png}
        \caption{ \name \ sections through first unconditional realization of Gaussian values generated by \texttt{NMROPT}. Objective components are calculated using these realizations. }
        \label{fig:nmropt_uncond_real0_\sec}
    \end{figure}
}

A critical component of the \texttt{NMROPT} workflow is checking the reproduction of the optimization targets. Deviations from these targets will likely manifest in the imputed conditioning data and final realizations.  Figures \ref{fig:nmropt_repro_vario}, \ref{fig:nmropt_repro_ivario}, \ref{fig:nmropt_repro_runs} show variogram, indicator variogram and cumulative run-length frequency reproduction with the optimal \gls{NMR} parameters. Variogram and indicator variogram reproduction is reasonable though there is some deviation from the target model, particularly at shorter ranges in the principal directions. This deviation comes at the expense of increasing high-grade (0.9 quantile) continuity in the direction of drilling (roughly parallel to the tertiary variogram direction). This increased continuity is evident in the 0.9 quantile run-length frequency plot (Figure \ref{fig:nmropt_repro_runs} right) where the unconditional \texttt{NMROPT} realizations show continuity greater than that in the data. As a consequence the 0.1 and 0.5 quantiles show slightly decreased continuity relative to the data (Figure \ref{fig:nmropt_repro_runs} left and center). This increased continuity is advantageous as results show the effect of conditioning in the final realizations can reduce extreme value continuity.

\begin{figure}[htb!]
    \centering
    \includegraphics[scale=0.75, max size={\textwidth}{\textheight}]{./0-Figures/06-Ch6/repro_vario.png}
    \caption{\texttt{NMROPT} continuous variogram reproduction for \csnreals \ realizations. Left to right are the major, minor and vertical directions, respectively.}
    \label{fig:nmropt_repro_vario}
\end{figure}

\begin{figure}
    \begin{subfigure}{1.0\textwidth}
        \centering
        \includegraphics[scale=0.75, max size={\textwidth}{\textheight}]{./0-Figures/06-Ch6/repro_ivario_0_1.png}
        \caption{0.1 Quantile}
    \end{subfigure}
    \begin{subfigure}{1.0\textwidth}
        \centering
        \includegraphics[scale=0.75, max size={\textwidth}{\textheight}]{./0-Figures/06-Ch6/repro_ivario_0_5.png}
        \caption{0.5 Quantile}
    \end{subfigure}
    \begin{subfigure}{1.0\textwidth}
        \centering
        \includegraphics[scale=0.75, max size={\textwidth}{\textheight}]{./0-Figures/06-Ch6/repro_ivario_0_9.png}
        \caption{0.9 Quantile}
    \end{subfigure}
    \caption{\texttt{NMROPT} indicator variogram reproduction for \csnreals \ realizations for the 0.1 (a), 0.5 (b) and 0.9 (c) quantiles. Left to right are the major, minor and vertical directions, respectively.}
    \label{fig:nmropt_repro_ivario}
\end{figure}

\begin{figure}[htb!]
    \centering
    \includegraphics[scale=0.75, max size={\textwidth}{\textheight}]{./0-Figures/06-Ch6/repro_runs.png}
    \caption{\texttt{NMROPT} cumulative run-length frequency reproduction for \csnreals \ realizations. Thresholds 1, 2, and 3 correspond to the 0.1, 0.5 and 0.9 quantiles, respectively.}
    \label{fig:nmropt_repro_runs}
\end{figure}


\FloatBarrier
\section{Latent Factor Imputation}
\label{sec:fact_imp}

The optimized \gls{NMR} parameters from Section \ref{sec:param} are the basis for latent factor imputation. The optimized parameters and the Gaussian pool are input to the program \texttt{NMRIMP} and realizations of latent factors are computed via \gls{SGRI}. \csnreals \ realizations are imputed for each factor specified in the Gaussian pool, plus the nugget effect. All imputed realizations are checked to ensure they (1) reproduce the data within the specified polishing tolerance (Table \ref{tab:imp_pars}); (2) imputed realizations are on average univariate Gaussian; (3) imputed factors are on average uncorrelated; and (4) each imputed factor on average reproduces its single structure variogram model.

\begin{table}[!htb]
    \centering
    \caption{\texttt{NMRIMP} parameters.}
    \resizebox{0.9\width}{!}{\begin{tabular}{cc}
\toprule
\textbf{Parameter} &    Value \\
\midrule
\textbf{Max nodes               } &    40 \\
\textbf{Max iterations: coarse  } & 25000 \\
\textbf{Max iterations: polish  } & 15000 \\
\textbf{Reject tolerance: coarse} &     0.10 \\
\textbf{Reject tolerance: polish} &     0.01 \\
\textbf{Constraint quantile     } &     0.90 \\
\textbf{Exclusion radius        } &    40 \\
\bottomrule
\end{tabular}
}
    \label{tab:imp_pars}
\end{table}

Figure \ref{fig:imp_data_scatter} shows the scatter plot of hard data values and the corresponding imputed values in normal score units. The imputed values in this context are the five latent factors mixed through the optimal \gls{NMR} and normal score transformed. There is roughly perfect correlation between data and imputed values and no delta values exceed the specified polishing tolerance. Figure \ref{fig:factor_cdfs} shows that each imputed factor is on average univariate Gaussian and Figure \ref{fig:fact_corrmat} shows the imputed factors are effectively uncorrelated. The left matrix is randomly generated, independent Gaussian data as a reference, and the right matrix is the average correlation between imputed values over \csnreals \ realizations. Figure \ref{fig:nmrimp_repro_gvario} shows variogram reproduction for each imputed factor. There is some deviation from the specified target models though overall, variogram reproduction is reasonable. This is particularly true for factors three and four where the range in the major direction is long relative to the domain size in that orientation.

\begin{figure}[htb!]
    \centering
    \includegraphics[scale=0.75, max size={\textwidth}{\textheight}]{./0-Figures/06-Ch6/imp_data_scatter.png}
    \caption{Scatter plot illustrating the relationship between the hard data values and imputed values in normal score units. }
    \label{fig:imp_data_scatter}
\end{figure}

\begin{figure}[htb!]
    \centering
    \includegraphics[scale=0.5, max size={\textwidth}{\textheight}]{./0-Figures/06-Ch6/factor_cdfs.png}
    \caption{\Gls{CDF} reproduction for the four latent factors. }
    \label{fig:factor_cdfs}
\end{figure}

\begin{figure}[htb!]
    \centering
    \includegraphics[scale=0.6, max size={\textwidth}{\textheight}]{./0-Figures/06-Ch6/fact_corrmat.png}
    \caption{Correlation matrices for collocated imputed latent factors. The left matrix is randomly generated Gaussian data as a reference, and the right matrix is the average correlation between imputed values over \csnreals \ realizations.}
    \label{fig:fact_corrmat}
\end{figure}

\begin{figure}
    \begin{subfigure}{1.0\textwidth}
        \centering
        \includegraphics[scale=0.75, max size={\textwidth}{\textheight}]{./0-Figures/06-Ch6/expvario_fact1.png}
        \caption{Factor 1}
    \end{subfigure}
    \begin{subfigure}{1.0\textwidth}
        \centering
        \includegraphics[scale=0.75, max size={\textwidth}{\textheight}]{./0-Figures/06-Ch6/expvario_fact2.png}
        \caption{Factor 2}
    \end{subfigure}
    \begin{subfigure}{1.0\textwidth}
        \centering
        \includegraphics[scale=0.75, max size={\textwidth}{\textheight}]{./0-Figures/06-Ch6/expvario_fact3.png}
        \caption{Factor 3}
    \end{subfigure}
    \begin{subfigure}{1.0\textwidth}
        \centering
        \includegraphics[scale=0.75, max size={\textwidth}{\textheight}]{./0-Figures/06-Ch6/expvario_fact4.png}
        \caption{Factor 4}
    \end{subfigure}
    \caption{Variogram reproduction for \csnreals \ imputed realizations of each factor. The variogram model is the model specified in the Gaussian pool in Table \ref{tab:pool}.}
    \label{fig:nmrimp_repro_gvario}
\end{figure}


\FloatBarrier
\section{Latent Factor Simulation}
\label{sec:fact_sim}

Imputed factor realizations from Section \ref{sec:fact_imp} become conditioning data for gridded factor realizations. Each univariate Gaussian factor realization is used to condition a gridded realization in a one-for-one fashion. That is, a unique data realization is used for each simulated realization. Gridded realizations in this case-study are simulated using \gls{SGS} though any conditional simulation algortithm is valid.

The point scale simulation grid utilizes a cell size of 2.5 x 2.5 x 5m for a total of 2,236,948 nodes; Table \ref{tab:simgrid} summarizes its parameters. \csnreals \ realizations of each factor are simulated and checked similar to any other Gaussian simulation workflow. The realizations are on average univariate Gaussian and reproduce the variogram. Figure \ref{fig:gridded_factors} shows a plan-view section through the first realization of the gridded factors.

\begin{table}[!htb]
    \centering
    \caption{Simulation grid parameters.}
    \resizebox{0.9\width}{!}{\begin{tabular}{cccc}
\toprule
{} & Easting & Northing & Elevation \\
\midrule
\textbf{minimum (m)} & -256.08 &  -227.04 &   -130.86 \\
\textbf{maximum (m)} &  308.92 &   277.96 &    114.14 \\
\textbf{size (m)   } &    2.50 &     2.50 &      5.00 \\
\textbf{number     } &     226 &      202 &        49 \\
\bottomrule
\end{tabular}
}
    \label{tab:simgrid}
\end{table}

\begin{figure}[htb!]
    \centering
    \includegraphics[scale=0.5, max size={\textwidth}{\textheight}]{./0-Figures/06-Ch6/gridded_factors.png}
    \caption{Plan-view section through the first realization of the gridded factors, excluding the nugget.}
    \label{fig:gridded_factors}
\end{figure}

Once factors are defined at all grid nodes they are mapped from latent to observed space through application of the optimized \gls{NMR}. Mapping consists of interpolating the raw activation values (\gls{NMR} output) with the normal score reference distribution generated by \texttt{NMRIMP}. Using a single reference distribution ensures consistency between imputation and gridded simulation. The \gls{NMR} realizations are checked ensuring they reproduce the continuous variogram model, indicator variogram models and cumulative run-length frequencies, as well as being on average univariate Gaussian. Figures \ref{fig:ns_reals_xy} to \ref{fig:ns_reals_yz} show sections through the first \gls{NMR} realization mapped to Gaussian units. Figure \ref{fig:repro_gridded_ns_cdf} shows \gls{CDF} reproduction of the \gls{NMR} realizations in Gaussian units. The realizations show reasonable variance given the level of conditioning and are on average univariate Gaussian. Figures \ref{fig:orig_reals_xy} to \ref{fig:orig_reals_yz} show sections through the first \gls{NMR} realization mapped to original units. These figures clearly show the vertical and downhole high-grade continuity imparted by factors three and four (Table \ref{tab:pool}). The overall background continuity dipping to the east imparted by the continuous variogram model is clear in the east-west sections (Figure \ref{fig:orig_reals_xz}). Figure \ref{fig:repro_gridded_cdf} shows \gls{CDF} reproduction of the \gls{NMR} realizations in back-transformed to units. The realizations reproduce the declustered histogram of the training data well.

\foreach \sec/\name in \sectuples
{
    \begin{figure}[htb!]
        \centering
        \includegraphics[scale=0.5, max size={\textwidth}{\textheight}]{./0-Figures/06-Ch6/ns_reals_\sec.png}
        \caption{ \name \ sections through the first \gls{NMR} realization mapped to Gaussian units. }
        \label{fig:ns_reals_\sec}
    \end{figure}
}

\begin{figure}[htb!]
    \centering
    \includegraphics[scale=0.75, max size={\textwidth}{\textheight}]{./0-Figures/06-Ch6/repro_gridded_ns_cdf.png}
    \caption{\gls{CDF} reproduction for \csnreals \ \gls{NMR} realizations mapped to Gaussian units. }
    \label{fig:repro_gridded_ns_cdf}
\end{figure}

\foreach \sec/\name in \sectuples
{
    \begin{figure}[htb!]
        \centering
        \includegraphics[scale=0.5, max size={\textwidth}{\textheight}]{./0-Figures/06-Ch6/orig_reals_\sec.png}
        \caption{ \name \ sections through the first \gls{NMR} realization back-transformed to original units. }
        \label{fig:orig_reals_\sec}
    \end{figure}
}

\begin{figure}[htb!]
    \centering
    \includegraphics[scale=0.75, max size={\textwidth}{\textheight}]{./0-Figures/06-Ch6/repro_gridded_cdf.png}
    \caption{ \gls{CDF} reproduction for \csnreals \ \gls{NMR} realizations back-transformed to original units. }
    \label{fig:repro_gridded_cdf}
\end{figure}


Figure \ref{fig:repro_gridded_vario} shows gridded continuous variogram reproduction for the \gls{NMR} realizations in Gaussian units. The experimental variogram of the first imputed data realization and the average of \gls{SGS} realizations are also shown for reference. There is reasonable agreement between the target model (blue), the imputed data and the gridded realization average. There is some deviation however these are the same deviations present from the inference of the network parameters (Figure \ref{fig:nmropt_repro_vario}). Indicator variogram reproduction in Figure \ref{fig:repro_gridded_ivario} shows similar deviations observed in the optimization process though the gridded variograms reproduce the target model and imputed data reasonably well. Figure \ref{fig:repro_gridded_runs} shows gridded cumulative run-length frequency reproduction for both the \gls{NMR} and \gls{SGS} realizations, respectively. The expected value of the \gls{SGS} realizations is shown in black. As expected the \gls{NMR} realizations show increased continuity over \gls{SGS}. \gls{SGS} shows greater continuity at shorter run-lengths however beyond a run-length of three composites, the \gls{NMR} model is more continuous.


\begin{figure}[htb!]
    \centering
    \includegraphics[scale=0.75, max size={\textwidth}{\textheight}]{./0-Figures/06-Ch6/repro_gridded_vario.png}
    \caption{\gls{NMR} continuous variogram reproduction for \csnreals \ realizations. Left to right are the major, minor and vertical directions, respectively.}
    \label{fig:repro_gridded_vario}
\end{figure}

\begin{figure}
    \begin{subfigure}{1.0\textwidth}
        \centering
        \includegraphics[scale=0.75, max size={\textwidth}{\textheight}]{./0-Figures/06-Ch6/repro_gridded_ivario_0_1.png}
        \caption{0.1 Quantile}
    \end{subfigure}
    \begin{subfigure}{1.0\textwidth}
        \centering
        \includegraphics[scale=0.75, max size={\textwidth}{\textheight}]{./0-Figures/06-Ch6/repro_gridded_ivario_0_5.png}
        \caption{0.5 Quantile}
    \end{subfigure}
    \begin{subfigure}{1.0\textwidth}
        \centering
        \includegraphics[scale=0.75, max size={\textwidth}{\textheight}]{./0-Figures/06-Ch6/repro_gridded_ivario_0_9.png}
        \caption{0.9 Quantile}
    \end{subfigure}
    \caption{\gls{NMR} indicator variogram reproduction for \csnreals \ realizations for the 0.1 (a), 0.5 (b) and 0.9 (c) quantiles. Left to right are the major, minor and vertical directions, respectively.}
    \label{fig:repro_gridded_ivario}
\end{figure}

\begin{figure}[htb!]
    \centering
    \includegraphics[scale=0.75, max size={\textwidth}{\textheight}]{./0-Figures/06-Ch6/repro_gridded_runs.png}
    \caption{\gls{NMR} 0.9 quantile cumulative run-length frequency reproduction for \csnreals \ realizations. 0.9 quantile from \gls{SGS} is shown for reference.}
    \label{fig:repro_gridded_runs}
\end{figure}


\FloatBarrier
\section{Post Processing}
\label{sec:post}

Point scale realizations are block-averaged to the \gls{SMU} volume for comparison at a relevant scale. The \gls{SMU} scale grid is centered on the south-west corner of the point scale grid that encompasses the majority of the test data set. Table \ref{tab:smugrid} shows the parameters of the \gls{SMU} grid. As a goal of the \gls{NMR} is to improve local high-grade resources conditional to certain events, a simplified stope optimization process is implemented to mimic the selection of local high-grade resources at the \gls{SMU} scale. A reference \gls{OK} model is built on the \gls{SMU} grid using the training data set and the original units variogram model of the training set. Stopes with a fixed footprint of 30 x 30 m and a minimum height of 40 m are selected considering the \gls{OK} reference model and a \gls{COG} of 2.0 units. An initial grid of stope footprint centroids is generated across the reference model. All valid footprints about the centroid are evaluated and the highest grade, minimum volume stope is selected. The initial stope's z-dimension is then expanded one grid level at a time until either the volume is no longer above \gls{COG} or a maximum height of 80 m is reached. This process is highly simplified and the resulting volumes are not true optimums. Figure \ref{fig:stopes_testdata_xy} shows a plan view section through the reference \gls{OK} model showing the stope shapes and the test data locations.

\begin{table}[!htb]
    \centering
    \caption{\Gls{SMU} grid parameters.}
    \resizebox{0.9\width}{!}{\begin{tabular}{cccc}
\toprule
{} & Easting & Northing & Elevation \\
\midrule
\textbf{minimum (m)} & -256.08 &  -227.04 &   -130.49 \\
\textbf{maximum (m)} &  128.92 &   127.96 &     -0.49 \\
\textbf{size (m)   } &    5.00 &     5.00 &      5.00 \\
\textbf{number     } &      77 &       71 &        26 \\
\bottomrule
\end{tabular}
}
    \label{tab:smugrid}
\end{table}

\begin{figure}[htb!]
    \centering
    \includegraphics[scale=0.75, max size={\textwidth}{\textheight}]{./0-Figures/06-Ch6/stopes_testdata_xy.png}
    \caption{Plan view section through the reference \gls{OK} model showing stope shapes and the test dataset. }
    \label{fig:stopes_testdata_xy}
\end{figure}

Stopes containing composites from the test dataset are used to evaluate the performance of the \gls{NMR} realizations relative to \gls{SGS} in local, high-grade regions of the model. The test composites are coded by stope ID and compared directly to simulated realizations within the same volume. The following section summarizes model performance.

\FloatBarrier
\section{Model Comparison}
\label{sec:compare}

The following section compares the \gls{NMR} and \gls{SGS} realizations conditional to high-grade stope volumes. Figure \ref{fig:nmr_sgs_gtcurve_all_stopes} shows \gls{GT} curves for all stope volumes for the \gls{NMR} (left) and \gls{SGS} (right) \gls{SMU} scale realizations. Test data tonnes and grade come from length weighted composites in the test set flagged within the stope volumes. This test shows how well each set of realizations reproduces data that the model has not seen. Over all stopes the \gls{SGS} model is both higher in grade and tonnage relative to the composites. \gls{NMR} realizations are overall lower in grade and tonnage relative to the \gls{SGS} realizations and more closely reproduce the test data. This is due to the anisotropies of the high-grade factors constraining high-grade features in particular directions, opposed to being controlled by a continuous variogram (that considers the full range of grade values) in the \gls{SGS} context.

\begin{figure}[htb!]
    \centering
    \includegraphics[scale=0.6, max size={\textwidth}{\textheight}]{./0-Figures/06-Ch6/nmr_sgs_gtcurve_all_stopes.png}
    \caption{\Gls{GT} curves for all stope volumes for the \gls{NMR} (left) and \gls{SGS} (right) models. }
    \label{fig:nmr_sgs_gtcurve_all_stopes}
\end{figure}

Table \ref{tab:nmr_sgs_stope_delta} shows more detailed grade and deviation data on a stope-by-stope basis. The table shows the number of test data in each stope, the mean grade of the composites, and the percent deviation from the test data mean for the \gls{OK}, \gls{SGS} and \gls{NMR} expected values. Values are sorted in
ascending order by \gls{NMR} expected value. There are many stopes where all models perform poorly, both over and underestimating the test data mean. These large deviations occur in stopes where the training and testing data are unbalanced. That is there is significant deviation between the mean of the conditioning data and the mean of the hold-out data. Where the training and testing data are more balanced, the \gls{NMR} model performs well relative to \gls{SGS}, particularly stopes 81, 560, 38 and 44. The \gls{NMR} shows deviations less than 3.3\% from the test data for these stopes while the \gls{SGS} model shows deviations from 9.6-14.9\%. It should be noted that there are stopes where \gls{SGS} does outperform the \gls{NMR}, (particularly when the mean of the test data increases above \gls{COG}), and stopes where neither model performs well.

\begin{table}[!htb]
    \centering
    \caption{\% difference in expected value between test data, \gls{OK} mean, \gls{SGS} expected value and \gls{NMR} expected value by stope (>2.0 unit \gls{COG}). Stopes with greater than 15 test data are shown. Values are sorted in ascending order by \gls{NMR} expected value.}
    \resizebox{0.9\width}{!}{\begin{tabular}{c*{6}{p{1.5cm}}}
\toprule
{} & Num Test Data & Train Data Mean & Test Data Mean & OK Mean ($\pm\%$) & SGS Exp ($\pm\%$) & NMR Exp ($\pm\%$) \\
\textbf{Stope ID} &               &                 &                &                   &                   &                   \\
\midrule
\textbf{341     } &            40 &            1.85 &           1.86 &               7.7 &              15.2 &               4.4 \\
\textbf{1138    } &            39 &            1.95 &           2.32 &             -13.2 &              -8.8 &             -12.0 \\
\textbf{46      } &            40 &            1.99 &           2.08 &               2.6 &               5.9 &              -3.1 \\
\textbf{1164    } &            30 &            2.06 &           1.76 &              15.3 &              27.8 &              20.3 \\
\textbf{157     } &            52 &            2.08 &           1.95 &               8.6 &               8.3 &               2.9 \\
\textbf{115     } &            73 &            2.15 &           1.98 &               1.6 &               1.7 &               1.4 \\
\textbf{47      } &            31 &            2.20 &           2.42 &              -6.9 &              -6.0 &              -8.9 \\
\textbf{461     } &            31 &            2.29 &           2.49 &             -15.8 &             -13.7 &             -15.1 \\
\textbf{40      } &            33 &            2.34 &           2.00 &              10.5 &              14.8 &              12.9 \\
\textbf{158     } &            33 &            2.56 &           2.19 &               3.9 &               2.8 &               0.4 \\
\textbf{39      } &            38 &            2.72 &           2.31 &               8.1 &              12.4 &               5.2 \\
\bottomrule
\end{tabular}
}
    \label{tab:nmr_sgs_stope_delta}
\end{table}

Figure \ref{fig:nmr_sgs_gtcurve_stope81} shows the \gls{NMR} vs \gls{SGS} \gls{GT} curve for stope 81. These \gls{GT} curves show similar characteristics to Figure \ref{fig:nmr_sgs_gtcurve_all_stopes} where the \gls{SGS} model is higher grade through virtually all cutoff grades. The \gls{NMR} model shows better reproduction of the test composites grades (below 2.5 units) and the fraction of total tonnes above \gls{COG}. Figure \ref{fig:stope81_test_data} shows east-west sections though the spatial centroid of stope 81 showing the reference \gls{OK} model, and the first realization of the \gls{SGS} and \gls{NMR} models with the test data. Visually the \gls{NMR} realization does a good job of reproducing the test data and vertical high-grade continuity imparted by factor three is apparent. Figure \ref{fig:stope310_test_data} shows sections through the volume of stope 310 where all models perform poorly. This discrepancy is due to a large deviation in grades between the training and testing data within the stope volume. The stope is centered on high-grade composites present in the training data however the mean of the test composites is below \gls{COG}.

% good stope
\begin{figure}[htb!]
    \centering
    \includegraphics[scale=0.6, max size={\textwidth}{\textheight}]{./0-Figures/06-Ch6/nmr_sgs_gtcurve_stope81.png}
    \caption{\Gls{GT} curves for stope 81 for \gls{NMR} (left) and \gls{SGS} (right) models. }
    \label{fig:nmr_sgs_gtcurve_stope81}
\end{figure}

\begin{figure}[htb!]
    \centering
    \includegraphics[scale=0.6, max size={\textwidth}{\textheight}]{./0-Figures/06-Ch6/stope81_test_data.png}
    \caption{Sections through stope 81 showing the test data, \gls{OK} model, and the first block averaged realization of the \gls{SGS} and \gls{NMR} models.}
    \label{fig:stope81_test_data}
\end{figure}

% % bad stope
% \begin{figure}[htb!]
%     \centering
%     \includegraphics[scale=0.6, max size={\textwidth}{\textheight}]{./0-Figures/06-Ch6/nmr_sgs_gtcurve_stope310.png}
%     \caption{\Gls{GT} curves for stope 310 for \gls{NMR} (left) and \gls{SGS} (right) models. }
%     \label{fig:nmr_sgs_gtcurve_stope310}
% \end{figure}

\begin{figure}[htb!]
    \centering
    \includegraphics[scale=0.6, max size={\textwidth}{\textheight}]{./0-Figures/06-Ch6/stope310_test_data.png}
    \caption{Sections through stope 310 showing the test data, \gls{OK} model, and the first block averaged realization of the \gls{SGS} and \gls{NMR} models.}
    \label{fig:stope310_test_data}
\end{figure}



\FloatBarrier
\section{Discussion}
\label{sec:dicuss06}

The \gls{NMR} shows the ability to reproduce high-grade continuity features that cannot be captured with a continuous variogram alone. The specification of $\omega$ bounds allows the practitioner to impart specific spatial features onto specific portions of the continuous grade range. The resulting \gls{NMR} realizations are univariate Gaussian, however they capture non-Gaussian connectivity features. In the context of this case-study, the connectivity of extreme values is of particular importance. A base-case \gls{SGS} model shows that realizations generated under a multivariate-Gaussian assumption cannot alone capture the high-grade continuity observed in the drillholes. The \gls{NMR} parameter inference workflow determines the optimal weight to each factor as well as the exponent $\omega$ for the network activation function. Parameters are optimized such that continuous, and indicator variograms and cumulative run-length frequencies observed in the drillhole data are reproduced in the final gridded realizations. The indicator variograms do not show significant asymmetry in this case-study though the \gls{NMR} possesses the ability to reproduce these non-Gaussian features. Addition high-grade continuity, above what is observed in the drillholes, is embedded in the \gls{NMR} through highly anisotropic factors where $\omega \ge 1$. This increased continuity seems to be required in some circumstances where the conditional simulation of the Gaussian factors leads to destructuring of extreme values even after being passed through the \gls{NMR}.

Over all stope volumes, the \gls{NMR} realizations more accurately reproduce the grade and tonnage of the test dataset than the \gls{SGS} realizations. On a stope-by-stope basis the performance of the \gls{NMR} model is variable though major deviations in grade between the train and test composites within any given stope should be considered. Stopes with many test data such as stope 81, the \gls{NMR} expected values shows a 9.4\% improvement over the \gls{SGS} expected value. Comparison of composite averages to stope volume expected values is assumed to be an adequate proxy to production reconciliation data; this ``test'' dataset scenario is likely as close as we can get without having true as-mined shapes and production grade data. It is also important to note that the stope optimization workflow is not generating a true optimal shape. The stope shapes are simply a geotechnically feasible heuristic proxy to mimic the mining selectivity.


