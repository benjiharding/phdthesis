%!TEX root = ../Thesis_example_compile_this.tex

%%%%%%%%%%%%%%%%%%%%%%%%%%%%%%%%%%%%%%%%%%%%%%%%%%%%%%%%%%%%%%%%%%%%%%%%%%%%%%%
\chapter{Introduction}
\label{ch:01intro}
%%%%%%%%%%%%%%%%%%%%%%%%%%%%%%%%%%%%%%%%%%%%%%%%%%%%%%%%%%%%%%%%%%%%%%%%%%%%%%%

\FloatBarrier
\section{Problem Statement}
\label{sec:01problem}

\FloatBarrier
\section{Literature Review}
\label{sec:01litreview}

\FloatBarrier
\subsection{Inverse Problems}
\label{subsec:03inverse}

Inverse problems encompass a broad class of problems where the objective is to infer the underlying causes or parameters of a system from observed data or measurable outputs. The prediction of a response is a forward problem, while the use of a response, or observed measurements to infer the properties of a model is an inverse problem \citep{tarantola2005inverse}. Inverse problems arise in various scientific disciplines, including physics, engineering, geosciences, medical imaging, and more. Geospatial inverse problems are common in both the fields of geophysics \citep{linde2015geological} and hydrogeology \citep{zhou2014inverse} where the underlying geologic model is unknown, however a set of measured responses, such as density or seismic properties, are known. The inverse problem involves inferring interpretable geologic properties of the unknown model, such as lithology or porosity, that satisfy the observed measurements. The process of solving inverse problems involves constructing a mathematical forward model that describes the relationship between the unknown parameters and the observed data; then using this model to infer the unknown parameters. A challenge of inverse problems is ill-posedness, or the lack of unique solution \citep{tarantola2005inverse}. It is possible that there are multiple (or infinite) solutions that are valid given the observed data. As an exact solution is rarely possible in natural, non-linear systems, one looks for possible solutions that are close to actual observations \citep{bardossy2016random}. For this reason many inverse problems are framed as an optimization problem, minimizing an objective function relevant to the problem at hand \citep{giraud2019integration,nava-flores2023high,athens2022stochastic}. The objective function is minimized iteratively. A forward modeling operator predicts an outcome for the current state of the model parameters and the objective function evaluates the loss between this prediction and the observed measurements. The optimization algorithm updates the parameter vector until it achieves an acceptable match between the model output and observed measurements.

Due to the inherent complexity and potential non-uniqueness of inverse problems, careful consideration of data quality, noise, and computational stability is essential to obtain meaningful and reliable solutions.

\begin{enumerate}
    \item something about the forward problem
    \item prior knowledge is the specification of the pool
    \item prior knowledge is the conceptual geologic model
    \item no data reproduction at this point, only reproduction of pertinent statistics
    \item constraints on wt magnitude, nscore keeps solution reasonable
\end{enumerate}

\FloatBarrier
\section{Notation and Concepts}
\label{sec:01notation}

\FloatBarrier
\section{Thesis Statement}
\label{sec:01thesis}