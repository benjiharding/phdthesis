%!TEX root = ../Thesis_example_compile_this.tex

\chapter*{Abstract}
\markboth{Abstract}{}

% Set-up double spacing for the abstract
\begin{DoubleSpace}
    \linespread{2.0}\selectfont{}

    Geostatistical models are often generated with widely spaced data configurations. Data collection costs prohibit exhaustive sampling and necessitate statistical inference from limited samples. Spatial prediction with sparse data in the presence of extreme values is an enduring challenge in the mining industry. Extreme values may have significant local influence, leading to overstated resources and the risk of production shortfalls. Practitioners are presented with difficult decisions for limiting extreme value influence and characterizing their spatial continuity. Inputs to numerical geologic models consist of observed data, a representative histogram and spatial controls on mineralization. Each of these components presents challenges in the presence of extreme values. Extreme values are often under-sampled, making inferences about their probability of occurrence difficult. The influence of extreme values is often limited in practice through grade capping, which could significantly impact the final resource. Extreme values' spatial continuity often differs from the barren or mineralized background. Traditional estimation and simulation methodologies are limited in adapting to extreme values and asymmetric spatial continuity features. These challenges motivate the development of a framework for the simulation of continuous variables with explicit consideration of high-order extreme value features. The proposed \acrfull{NMR} framework constructs a continuous \gls{RV} as a non-linear mixture of latent Gaussian factors and does not require capping or modification of extreme grade values. Mixing parameters are inferred via optimization, considering two- and multi-point connectivity features at distinct indicator thresholds. The multi-component objective function permits the reproduction of high-order connectivity features and asymmetric indicator spatial continuity that cannot be captured by a single Gaussian \gls{RF} model. The latent Gaussian factors are imputed such that they exactly reproduce the observed data values when mixed. The applicability of the proposed methodology is demonstrated on a mineral deposit where the project operators note non-Gaussian, extreme value features in drillhole data.


\end{DoubleSpace}
